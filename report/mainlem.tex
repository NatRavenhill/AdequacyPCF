\begin{proof}
By induction on the possible values of $e$.

\paragraph{Variables} For a variable $x_1 : A_1, \dots, x_n : A_n \vdash x : A$, assume for any $i = 1, \dots, n$ that $d_i R_{A_i} e_i$. Then we want to show:

\[ (\lambda (d_1, \dots, d_n) \in \llbracket \Gamma \rrbracket. \ \pi_i(d_1, \dots, d_n))(d_1, \dots, d_n) \ R_A \ [e_1/x_1, \dots, e_n/x_n]x \]

As we have $\Gamma \vdash x : A$, by the typing rule for variables we have $\Gamma(x) = A$, so $x \in dom(\Gamma)$ and $\exists i. \ d_i R_{A_i} e_i$ . Then on the right hand side we have $[e_i/x]x$. Therefore we want to show

\[ d_i R_{A_i} e_i \]

which we have as an assumption. 

%Therefore we know there is some $e'$ such that $e'/x \in \gamma$ and by our assumption there is some $d$ such that $d R_A e'$.

%On the right hand side, the substitution $[\gamma]$ will apply the substitution for $x$ and ignore everything else in $\gamma$. Therefore we want to show $d R_A e'$, which we already have as an assumption.

\paragraph{Zero} For $z : \nat$, we want to show:

\[ \llbracket \Gamma \vdash z : \nat \rrbracket (d_1, \dots, d_n) \ R_{\nat} \ [e_1/x_1, \dots, e_n/x_n]z \]

Expanding the definitions gives us $0 \ R_{\nat} \ z$, so we must show $\forall n \in \mathbbm{N}. 0 = n \Rightarrow z \mapsto^* n$, which is the case as $z$ reduces to $n$ in zero steps. Therefore $0 \ R_{\nat} \ z$.

\paragraph{Successor} For $x_1 : A_1, \dots, x_n : A_n \vdash s(e) : \nat$, assume for any $i = 1, \dots, n$ that $d_i R_{A_i} e_i$. Then we want to show:

\[ \llbracket \Gamma \vdash s(e) : \nat \rrbracket (d_1, \dots, d_n) \ R_{\nat} \ [e_1/x_1, \dots, e_n/x_n](s(e))\]

Which expands to $\forall n \in \mathbbm{N}. \ \llbracket \Gamma \vdash s(e) : \nat \rrbracket (d_1, \dots, d_n) = n \Rightarrow [e_1/x_1, \dots, e_n/x_n]s(e) \mapsto^* n $. By the inductive hypothesis, we know:

\[ \llbracket \Gamma \vdash e : \nat \rrbracket (d_1, \dots, d_n) \ R_{\nat} \ [e_1/x_1, \dots, e_n/x_n]e \]

This expands to $\forall n \in \mathbbm{N}. \ \llbracket \Gamma \vdash e : \nat \rrbracket (d_1, \dots, d_n) = n \Rightarrow [e_1/x_1, \dots, e_n/x_n]e \mapsto^* n $. (If the denotation of $e$ is $\bot$ then this is vacuously true.)

Therefore there will be two cases:

\begin{enumerate}
\item{If $\llbracket \Gamma \vdash e : \nat \rrbracket (d_1, \dots, d_n) = \bot$, then we must show $\bot R_{\nat} [e_1/x_1, \dots, e_n/x_n]s(e)$, which we get from Lemma \ref{bot}}%As there is no $n = \bot$ then the statement that gives $\bot R_{\nat} [x_1/t_1, \dots, x_n/t_n]s(e)$ is vacuously true.}
\item{If $\llbracket \Gamma \vdash e : \nat \rrbracket (d_1, \dots, d_n) = v$, then we must show $v+1 \ R_{\nat} \ [e_1/x_1, \dots, e_n/x_n]s(e)$. Let $n = v + 1$. From the inductive hypothesis we know that $[e_1/x_1, \dots, e_n/x_n]e \mapsto^* v$. Using the congruence evaluation rule for successor, we get $s([e_1/x_1, \dots, e_n/x_n]e) \mapsto s(v)$ and $s(v)$ is the same as $v+1$. Therefore we have $v + 1 \ R_{Nat} s(v)$, so if we use this with the congruence rule in Lemma \ref{exp}, we have  $v + 1 \ R_{Nat} s([e_1/x_1, \dots, e_n/x_n]e)$ }
\end{enumerate}

\paragraph{Case} For $x_1 : A_1, \dots, x_n : A_n \vdash \case (e,z \mapsto e_0, s(x) \mapsto e_S)$, assume for any $i = 1, \dots, n$ that $d_i R_{A_i} e_i$. Then we want to show:


\[ \llbracket \Gamma \vdash \case(e,z \mapsto e_0, s(x) \mapsto e_S) : A \rrbracket (d_1, \dots, d_n) \]
\[ R_A \]
\[ [e_1/x_1, \dots, e_n/x_n]\case (z \mapsto e_0, s(x) \mapsto e_S) \]

The result of the denotation depends on the value of $e$, so we have three cases:

\begin{enumerate}
\item{$\llbracket \Gamma \vdash e : \nat \rrbracket (d_1, \dots, d_n) = \bot$, then we must show $\bot \ R_A \ [e_1/x_1, \dots, e_n/x_n] \\ \case (z \mapsto e_0, s(x) \mapsto e_S)$, which we get by applying Lemma \ref{bot}.}
\item{$\llbracket \Gamma \vdash e : \nat \rrbracket (d_1, \dots, d_n) = 0$, then we want to show $\llbracket \Gamma \vdash e_0 : A \rrbracket (d_1, \dots, d_n) \ \\ R_A [e_1/x_1, \dots, e_n/x_n]\case (z \mapsto e_0, s(x) \mapsto e_S)$. As we have $\Gamma \vdash e_0 : A$ from the typing rule of case, we can get $\llbracket \Gamma \vdash e_0 : A \rrbracket (d_1, \dots, d_n) \ R_A [e_1/x_1, \dots, e_n/x_n]e_0$ by the induction hypothesis. We can now use this in Lemma \ref{exp}, with the operational semantics rule for case when the expression is zero, to get 

\[\llbracket \Gamma \vdash e_0 : A \rrbracket (d_1, \dots, d_n) \ R_A [e_1/x_1, \dots, e_n/x_n]\case (z \mapsto e_0, s(x) \mapsto e_S)\]
}
\item{$\llbracket \Gamma \vdash e : \nat \rrbracket (d_1, \dots, d_n) = n+1$ we want to show
\[ \llbracket \Gamma, x : \nat \vdash e_S : \nat \rrbracket(d_1, \dots , d_n, d) \]
\[\ R_A \]
\[ [e_1/x_1, \dots, e_n/x_n]\case (e, z \mapsto e_0, s(x) \mapsto e_S)\]

From the induction hypothesis we have 

\[ \llbracket \Gamma, x : \nat \vdash e_S : \nat \rrbracket(d_1, \dots, d_n, d) \ \ R_A \ [e_1/x_1, \dots, e_n/x_n, e/x]e_S\]

We can now use this in Lemma \ref{exp}, with the operational semantics rule for case when the expression is not zero to get 

\[ \llbracket \Gamma, x : \nat \vdash e_S : \nat \rrbracket(d_1, \dots , d_n, d) \]
\[ R_A \]
\[ [e_1/x_1, \dots, e_n/x_n]\case (e, z \mapsto e_0, s(x) \mapsto e_S)\].


}
\end{enumerate}

\paragraph{Application} For $x_1 : A_1, \dots, x_n : A_n \vdash e \ e' : B$, assume for any $i = 1, \dots, n$ that $d_i R_{A_i} e_i$. Then we want to show:

\[ \llbracket \Gamma \vdash  e \ e' : B \rrbracket (d_1, \dots, d_n) \]
\[ R_B \]
\[ [e_1/x_1, \dots, e_n/x_n]e \ e' \]

Using the denotational semantics and substitution function we can rewrite this as:

\[ \llbracket \Gamma \vdash  e : A \to B \rrbracket (d_1, \dots, d_n) (\llbracket \Gamma \vdash e' : B \rrbracket (d_1, \dots, d_n))\]
\[ \ R_B \]
\[ ([e_1/x_1, \dots, e_n/x_n]e) ([e_1/x_1, \dots, e_n/x_n] e') \]

By the inductive hypothesis we have

\[\llbracket \Gamma \vdash  e : A \to B \rrbracket (d_1, \dots, d_n) \ R_{A \to B} \ [e_1/x_1, \dots, e_n/x_n]e\]


Expanding this gives us $ \forall d \in \llbracket A \rrbracket. \ \forall e' \in \{ e \ | \ \vdash e : A\}. \ d R_A e' \Rightarrow \llbracket \Gamma \vdash  e : A \to B \rrbracket (d_1, \dots, d_n) (d) \  R_B \ [e_1/x_1, \dots, e_n/x_n]e \ e'$. Let $d = \llbracket \Gamma \vdash e' : B \rrbracket (d_1, \dots, d_n)$ and $e' = [e_1/x_1, \dots, e_n/x_n]e'$. Then we have 

\[\llbracket \Gamma \vdash  e : A \to B \rrbracket (d_1, \dots, d_n) (\llbracket \Gamma \vdash e' : B \rrbracket (d_1, \dots, d_n) \ R_B \ ([e_1/x_1, \dots, e_n/x_n]e) ([e_1/x_1, \dots, e_n/x_n] e')\]

\paragraph{$\lambda$-Abstraction} 
For $x_1 : A_1, \dots, x_n : A_n \vdash \lambda x : A. \ e : A \to B$, assume for any $i = 1, \dots, n$ that $d_i R_{A_i} e_i$. Then we want to show:

\[ \llbracket \Gamma \vdash \lambda x : A. \ e : A \to B \rrbracket (d_1, \dots, d_n) \ R_{A \to B} \ [e_1/x_1, \dots, e_n/x_n] (\lambda x : A. \ e) \]


%\[ \lambda a \in \llbracket A \rrbracket. \ \llbracket \Gamma, x : A \vdash e : B \rrbracket(\gamma, a/x) \ R_{A \to B} [\gamma] (\lambda x : A. \ e) \]

Expanding the definition of the logical relation gives us

\[ \forall d \in \llbracket A \rrbracket. \ \forall e' \in \{ e \ | \ \vdash e : A\}. \ d R_A e' \Rightarrow\]
\[ \llbracket \Gamma ,x:A \vdash e : B \rrbracket(d_1, \dots, d_n) (d) \ R_B \  [x_1/t_1, \dots, x_n/t_n](\lambda x:A. \ e) \ e'\]

Let $d$ be an element of the domain of type $A$ and $e'$ be an expression of type $A$ such that $d R_A  e'$.

Then by the using the denotational semantics for $\lambda$ abstraction, we want to show:

\[ (\lambda d \in \llbracket A \rrbracket. \ \llbracket \Gamma, x :A \vdash e : B \rrbracket (d_1, \dots, d_n,d))(d) \ R_B \ [e_1/x_1, \dots, e_n/x_n] (\lambda x:A. \ e) \ e'\]

which is the same as:

\[ \llbracket \Gamma, x :A \vdash e : B \rrbracket (d_1, \dots, d_n, d) \ R_B \ [e_1/x_1, \dots, e_n/x_n, e'/x] \ e \]

which we get by the inductive hypothesis, because from the evaluation rule we have \[[e_1/x_1, \dots, e_n/x_n] (\lambda x:A. \ e) \ e'  \mapsto [e'/x][e_1/x_1, \dots, e_n/x_n]e = [e_1/x_1, \dots, e_n/x_n, e'/x]e\]

 so we can use Lemma \ref{exp}. 

\paragraph{Fixpoint}
For $x_1 : A_1, \dots, x_n : A_n \vdash \fix x:A. \ e : A$, assume for any $i = 1, \dots, n$ that $d_i R_{A_i} e_i$. Then we want to show:

\[ \llbracket \Gamma \vdash \fix x:A. \ e : A \rrbracket (d_1, \dots, d_n) \ R_A \ [e_1/x_1, \dots, e_n/x_n] (\fix x:A. \ e : A) \]

The denotation of the left hand side is $fix( \lambda a \in \llbracket A \rrbracket. \  \llbracket \Gamma, x :A \vdash e : A \rrbracket(d_1, \dots, d_n, a))$.

By the fixpoint theorem, we know this is the same as 
\[\bigsqcup_n (\lambda a \in \llbracket A \rrbracket. \  \llbracket \Gamma, x :A \vdash e : A \rrbracket(d_1, \dots, d_n, a))^n(\bot)\]


If we can prove that \[\forall n \in \mathbbm{N}. \ (\lambda a \in \llbracket A \rrbracket. \ \llbracket \Gamma, x : A \vdash \ e : A \rrbracket(d_1 \dots d_n, a))^n(\bot) \ R_A \ [e_1/x_1, \dots e_n/x_n]\fix x : A. \ e : A\]

then we can use Lemma \ref{chain}, to get the above result.

We prove this by induction on $n$. When $n = 0$, we need to show $\bot \ R_A \ [e_1/x_1, \dots e_n/x_n]\fix x : A. \ e : A$, for which we just use Lemma \ref{bot}.

For the inductive case, we must show 

\[(\lambda a \in \llbracket A \rrbracket. \ \llbracket \Gamma, x : A \vdash e : A \rrbracket(d_1 \dots d_n,a))^{n+1}(\bot) \ R_A \ [e_1/x_1, \dots e_n/x_n]\fix x : A. \ e : A\]

We can rewrite the left hand side as $(\lambda a \in \llbracket A \rrbracket. \ \llbracket \Gamma, x : A \vdash e : A \rrbracket(d_1 \dots d_n,a))(\lambda a \in \llbracket A \rrbracket. \ \llbracket \Gamma, x : A \vdash e : A \rrbracket(d_1 \dots d_n,a))^n(\bot)) $, which is the same as:

\[ \llbracket \Gamma, x : A \vdash e : A \rrbracket(d_1 \dots d_n,(\lambda a \in \llbracket A \rrbracket. \ \llbracket \Gamma, x : A \vdash e : A \rrbracket(d_1 \dots d_n,a))^n(\bot)))\]

We know that 

\[(\lambda a \in \llbracket A \rrbracket. \ \llbracket \Gamma, x : A \vdash e : A \rrbracket(d_1 \dots d_n,a))^n(\bot))) \ R_A \ [e_1/x_1, \dots e_n/x_n]\fix x : A. \ e : A\]

by the inductive hypothesis, so we can use this as an assumption in the inductive hypothesis of the Main Lemma to get:

\[ \llbracket \Gamma, x : A \vdash e : A \rrbracket(d_1 \dots d_n,(\lambda a \in \llbracket A \rrbracket. \ \llbracket \Gamma, x : A \vdash e : A \rrbracket(d_1 \dots d_n,a))^n(\bot)))\]
\[ R_A \]
\[ [e_1/x_1, \dots e_n/x_n, [e_1/x_1, \dots e_n/x_n]\fix x : A. \ e : A/x]e \]

because 
\[[e_1/x_1, \dots e_n/x_n]\fix x : A. \ e : A \]

\[ = \fix  x : A. [e_1/x_1, \dots e_n/x_n]e : A \]

\[\mapsto [\fix x:A. [e_1/x_1, \dots e_n/x_n]e/x][e_1/x_1, \dots e_n/x_n] e : A\]

\[ = [[e_1/x_1, \dots e_n/x_n] \fix x:A. e/x] [e_1/x_1, \dots e_n/x_n] e : A\]

\[ = [e_1/x_1, \dots e_n/x_n, [e_1/x_1, \dots e_n/x_n] \fix x:A. e/x] e : A \]

We can use Lemma \ref{exp}, to get 
\[(\lambda a \in \llbracket A \rrbracket. \ \llbracket \Gamma, x : A \vdash e : A \rrbracket(d_1 \dots d_n,a))^{n+1}(\bot) \ R_A \ [e_1/x_1, \dots e_n/x_n]\fix x : A. \ e : A\].



%[\fix x:A. e : A/x][e_1/x_1, \dots e_n/x_n]\fix x : A. \ e : A $

%Then we use Lemma \ref{exp} with $[x_1/t_n, \dots, x_n/t_n] \fix x:A. \ e :A \mapsto 

 
%Using the fixpoint constant (see \ref{const}) , we can rewrite this as: 

%\[ \llbracket \Gamma \vdash \FIX \ (\lambda x:A. \ e) : A \rrbracket (d_1, \dots, d_n) \ R_A \ [x_1/t_1, \dots, x_n/t_n] (\FIX \ (\lambda x:A. \ e) : A) \]

%As $\FIX \ (\lambda x:A. \ e)$ is a function application, by its typing rule we have $\Gamma \vdash \lambda x:A . \ e : A \to A$, so we can use the inductive hypothesis to get:

%\[ \llbracket \Gamma \vdash \lambda x:A. \ e \rrbracket(d_1, \dots, d_n) \ R_{A \to A} \ [x_1/t_1, \dots, x_n/t_n](\lambda x :A. \ e)\]

%Then, we use Lemma \ref{fix} with $f = \llbracket \Gamma \vdash \lambda x : A . \ e \rrbracket (d_1, \dots, d_n)$ and $e = [x_1/t_1, \dots, x_n/t_n](\lambda x :A. e)$. This gives us $\llbracket \FIX \rrbracket \llbracket \Gamma \vdash \lambda x : A . \ e \rrbracket (d_1, \dots, d_n) \ R_A \ \FIX \ [x_1/t_1, \dots, x_n/t_n](\lambda x :A. e)$.

%Using the denotational semantics for function application, we have:

%\[ \llbracket \Gamma \vdash \FIX \ \lambda x : A . \ e \rrbracket (d_1, \dots, d_n) \ R_A \ [x_1/t_1, \dots, x_n/t_n]\FIX (\lambda x : A . \ e)\].




%Which we can rewrite as:
%
%\[ fix(\lambda a \in \llbracket A \rrbracket. \ \llbracket \Gamma, x:A \vdash e:A \rrbracket (\gamma, a/x)) \ R_A \ \fix x:A. \ [\gamma]e : A \]
%
%We want to use Lemma \ref{fix}, so we need to show:
%
%\textcolor{red}{\[ \lambda a \in \llbracket A \rrbracket. \ \llbracket \Gamma, x:A \vdash e:A \rrbracket (\gamma, a/x) \ R_{A \to A} \ [\gamma]e \]}
%
%\textcolor{red}{But we don't know that $[\gamma]e$ is a function!} 
%
%which expands to:
%
%\[\forall d \in \llbracket A \rrbracket. \ \forall e' \in \{ e \ | \ \vdash e : A\}. \ d R_A e' \Rightarrow \llbracket \Gamma, x:A \vdash e:A \rrbracket (\gamma, d/x) R_B  [\gamma]e \ e'\]
%
%The induction hypothesis is:
%
%\[ \llbracket \Gamma, x : A \vdash e : A \rrbracket(\gamma, d/x) \ R_A \  [\gamma, e'/x] e \]
%
%







\end{proof}