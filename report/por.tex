The following chapter is based on contents from the chapters "The Full Abstraction Problem" and "Logical Relations" from the book \citep{Streicher06}.

In Chapter \ref{ch6}, we described how our denotational model can represent any PCF expression. We can show that this model also contains extra functions that do not correspond to any expression. One such example is the function parallel or, written as $\por$.

Parallel or (por) is the function defined in the following way (where $0 = true$):

\[ \por \ x \ y =
\left\{
	\begin{array}{ll}
		0  & \mbox{if } x = 0  \vee y = 0 \\
		1 & \mbox{if } x = y = 1 \\
		\bot & \mbox{otherwise}
	\end{array}
\right.\]

which gives us the following truth table:
\begin{center}
\begin{tabular}{ c | c | c | c}
    & 0 & 1 & $\bot$ \\
    \hline
   0 & 0 & 0 & 0 \\
   1 & 0 & 1 & $\bot$ \\
   $\bot$ & 0 & $\bot$ & $\bot$\\   
\end{tabular}
\end{center}

\vspace{0.5cm}

By observation, parallel or is not definable as we would have to evaluate its arguments in parallel and PCF is a sequential language.

We can formally prove that $\por$ is non-definable by using logical relations.

\subsection{Logical Relation Definition}
We specify a logical relation on denotations of program terms in PCF  by defining relations on each type individually. We use Streicher's definition, as given in \citep{Streicher06}: 

\vspace{0.5cm}

\begin{defn}
Let $W$ be an arbitrary set.  A $W$-ary logical relation on the model of PCF is a family of relations

\[R = (R_A \in \mathcal{P}(\llbracket A \rrbracket^W) \ | \ A \in Type)\]

such that

\[f \in R_{A \to B} = \forall d \in R_A. \ \lambda i \in W. \ f(i)d(i) \in R_B\]
\end{defn}

where $Type$ is the set of all possible types our programs can have, defined by induction.

Logical relations for different properties are constructed by defining the relation differently on the base type for each property. For example, if the types are defined by the base type $\nat$ and other types $A \to B$ (formed any other types $A$ and $B$), then the relation is defined by the definition of $R_{\nat}$.

Therefore we say that a logical relation $R$ of arity $W$ is uniquely determined by $R_{\nat}$, so for all subsets of ${\llbracket \nat \rrbracket}^W$ there is a unique $R$ equal to the set. 

\vspace{0.5cm}

\paragraph{Logical Relations at Function Types}

For a function $f = (f_1, \dots f_n)$ to be in the relation, if we apply it to \textbf{any} value that is in the relation of the type of its domain, for example, for arity $3$:

\[ (x,y,z) \in R_A \]

Then $f$ applied to everything in these elements will be in the relation of the codomain, so we must have:

\[ (f_1(x),f_2(y),f_3(z)) \in R_B \]

Then $f \in R_{A \to B}$.

%Define a set of types, $Type$, which is formed by:
%
%\[ \theta ::= o \ | \ \theta \to \theta \]
%
%such that $\llbracket o \rrbracket = X$ and  $\llbracket \theta_1 \to \theta_2 \rrbracket = \llbracket \theta_1 \rrbracket \to \llbracket \theta_2 \rrbracket$
%
%\begin{defn}
%An $n$-ary \textbf{logical relation} is a family $\mathcal{R} = \{R_\theta\}_{\theta \in Type}$ of $n$-ary relations such that $R_\theta \subseteq \llbracket \theta \rrbracket \times \dots \llbracket \theta \rrbracket$ (i.e.\ an $n$-tuple) for any $\theta$ and
%
%\[ R_{\theta_1 \to \theta_2}(f_1, \dots f_n) \Leftrightarrow\]
%
%
%for all $(d_1, \dots d_n) \in \llbracket \theta_1 \rrbracket^n$, if $R_{\theta_1}(d_1, \dots d_n)$ then $R_{\theta_2}(f_1(d_1), \dots f_n(d_n))$ 
%\end{defn}

\subsection{Examples and Counter-Examples}

The examples in this section are our answers to  exercises from Lecture 1 of \citep{Murawski12}:

Applying certain restrictions on $R_{\nat}$ restricts the relations we can define on function types, for example:

\begin{itemize}
\item{If $R_{\nat} = \llbracket \nat \rrbracket^n$, relations on function types can only have $n$ arguments, if $n$ is specified, otherwise 
there is no difference to the definition.}
\item{If $R_{\nat} = \emptyset$, then all relations in $R$ are the empty relation.}
\end{itemize}

\paragraph{Union of two logical relations}
Given two logical relations $R$ and $S$ of the same arity, we could try to form a logical relation $R \cup S$. This is \textbf{not} a logical relation. For example, if we have $(f_1, \dots , f_n) \in R_{A \to B}$ (and not in $S_{A \to B}$) and $(d_1, \dots , d_n) \in S_A$, (and not in $R_{A \to B}$)  then $(f_1(d_n), \dots , f_n(d_n))$ cannot be in either $R$ or $S$.

\paragraph{Intersection of two logical relations}
Therefore if we restrict the logical relation to only contain tuples that are in both $R$ and $S$ then we do not have this problem, so this is a logical relation.

If the relations did not have the same arity, there will be no tuples that are in both relations that satisfy the definition, as if we have $R$ of arity 3 and $S$ of arity 5, then have, for example, $(f,g,h) \in R_{A \to B}$ and $(f,g,h,i,j) \in S_{A \to B}$, then we cannot say that $(f,g,h)$ is in both, as this would not be in $S$ anymore. Therefore, this would not be a logical relation.

\paragraph{Composition of two binary logical relations}
We define $R;S$ in the following way:

\[ (R;S)(x,y) \Leftrightarrow \exists z. \ (R(x,z) \ \wedge \ S(z,y))\]

So for base type we have:

\[ (R_{\nat};S_{\nat})(x,z) \Leftrightarrow \exists y. \ (R_{\nat}(x,y) \ \wedge \ S_{\nat}(y,z))\]

and for functions we have

\[ (R_{A \to B};S_{A \to B})(f,h) \Leftrightarrow \exists g. \ (R_{A \to B}(f,g) \ \wedge \ S_{A \to B}(g,h))\] 

where for any $(x,z) \in (R_A;S_A)$ we have

\[ (R_B(f(x),g(y)) \ \wedge \ S_B(g(y),h(z)))\] 

So as functions in $R$ are only applied to inputs that are in $R$ and the same for $S$, and functions in both are only applied to inputs that are in both, then this is a logical relation.

 

\subsection{Main Lemma}
When using logical relations, we usually prove a general theorem about the relation, which we then give specific inputs to, to get the proof of our original property. This is called the Main Lemma. %(Note that this definition uses the denotational semantics of PCF, so reading Chapter \ref{ch6} first will make this definition much easier to understand): 

\vspace{0.5cm}

For any denotation of a PCF term, we want to show that it is in a given logical relation at its type, so we want to show that $\llbracket \Gamma \vdash e : A \rrbracket \in R_{\Gamma \to A}$. An element of $\llbracket \Gamma \rrbracket$ is any tuple of substitutions $d^* = (d_1, \dots d_n)$ for $x_1 : A_1, \dots x_n :A_n = \Gamma$. So $R_\Gamma \in \mathcal{P}(\llbracket A_1 \rrbracket \times \dots \times \llbracket A_n \rrbracket$).

We want all substitutions in a set $W$ to be in the relation, so by using the definition of $\llbracket \Gamma \vdash e : A \rrbracket \in R_{\Gamma \to A}$, we want to show that 

\[ \forall (d^*(1), \dots d^*(W)) \in R_\Gamma. \ \lambda i \in W. \llbracket \Gamma \vdash e : A \rrbracket d^*(i)\]

This says that for any position in the $W$-tuple, we have the denotation of $e$ using the substitution $d^*(i)$ (the substitution from the $i$th position in the tuple $(d^*(1), \dots, d^*(W)) \in R_\Gamma$).

For $W$ different substitutions we want to have

\[(\llbracket \Gamma \vdash e : A \rrbracket d^*(1) , \dots, \llbracket \Gamma \vdash e : A \rrbracket d^*(W)) \in R_B \]

Therefore for program terms, we define the Main Lemma in the following way, as in \citep{Streicher06}, where we only need to define the lemma on $\lambda$ terms. This is because any closed PCF term can be written as a $\lambda$-term in the following way:

\begin{itemize}
\item{$z$}
\item{$\lambda x: \nat. \ s(x)$}
\item{$\lambda x: \nat, e_0 : A, e_S: A. \  \case(e,z \mapsto e_0,s(x) \mapsto e_S)$}
\item{$\lambda x : A. \ \lambda e: A . \ \fix x:A \ e $}
\end{itemize}

The statement of the Main Lemma is the following:

\vspace{0.5cm}

\begin{lem}\label{main2}
Let $R$ be a logical relation of arity $W$ on the Scott Model of PCF. Then for $\lambda$ terms $\Gamma \vdash e : A$ and $d_j \in R_{A_j}$ for $j = 1, \dots, n$

\[ \lambda i \in W. \llbracket \Gamma \vdash e : A \rrbracket d^*(i) \in R_A\]

where $d^*(i) = (d_1(i) \dots d_n(i))$ and $\Gamma = x_1 : A_1 , \dots x_n : A_n$
\end{lem}

We include the proof given by \citep{Streicher06}, additionally explaining how the case for variables is proved:

\begin{proof}
By induction on $\lambda$ terms:

\paragraph{Variables} Assume $x_1 : A_1, \dots , x_n : A_n \vdash x_j : A_j$ and $d_j \in R_{A_j}$ for every $j$. We need to show that $\lambda i \in W. \ \llbracket x_1 : A_1, \dots x_n : A_n \vdash x_j : A_j \rrbracket(d^*(i)) \in R_{A_j} $. $d^*(i)$ includes $d_j(i)$, so by the denotational semantics for variables, we want to show that $\lambda i \ \in W. d_j(i) \in R_{A_J}$. As we have assumed that $d_j \in R_{A_j}$, then at each $i$th position, we already have $d_j(i) \in R_{A_j}$ as an assumption, so we have  $\lambda i \ \in W. d_j(i) \in R_{A_j}$.
 
\paragraph{$\lambda$-Abstraction} Assume that $\Gamma \vdash \lambda x : A. \ e : B $ and $\forall j \leq n. \ d_j \in R_{A_j}$. We need to show that $\lambda i \in W. \  \llbracket \Gamma \vdash \lambda x : A. \ e : B \rrbracket (d^*(i)) \in R_{A \to B}$. As this is a function type, we expand the definition of the logical relation to get:

\[ \forall d \in R_A. \ \lambda i \in W. \ \llbracket \Gamma \vdash \lambda x:A. e : B \rrbracket (d^*(i))(d(i)) \in R_B\]

Let $d$ be a substitution in $R_A$. We use the induction hypothesis with $\Gamma, x : A \vdash e : B$ (obtained by using the typing rule of $\lambda$-abstraction) and $d_j \in R_{A_j}$  and $d \in R_A$ to get:

\[\lambda i \in W. \ \llbracket \Gamma ,x:A \vdash e : B \rrbracket (d^*(i),d(i))\in R_B \]

By the definition of the Denotational Semantics for $\lambda$-abstraction, this is the same as:

\[\lambda i \in W. \ \llbracket \Gamma \vdash \lambda x:A. e : B \rrbracket (d^*(i))(d(i)) \in R_B\]

Therefore, $\forall d \in R_A. \ \lambda i \in W. \ \llbracket \Gamma \vdash \lambda x:A. e : B \rrbracket (d^*(i))(d(i)) \in R_B$, so we have 

\[\lambda i \in W. \  \llbracket \Gamma \vdash \lambda x : A. \ e : B \rrbracket (d^*(i)) \in R_{A \to B}\]

which is what we wanted to prove.

\paragraph{Application} Assume that $\Gamma \vdash e \ e' :  B$ and $\forall j \leq n. \ d_j \in R_{A_j}$. We need to show that $\lambda i \in W. \  \llbracket \Gamma \vdash  e \ e' : B \rrbracket (d^*(i)) \in R_B$.

By using the induction hypothesis on $\Gamma \vdash e : A \to B$ and $\Gamma \vdash e' : A$ (which we obtain by using the typing rule of function application), we get 
$\lambda i \in W. \  \llbracket \Gamma \vdash e : A \to B \rrbracket(d^*(i)) \in R_{A \to B}$. and $\lambda i \in W. \ \llbracket \Gamma \vdash e' : A \rrbracket(d^*(i)) \in R_{A}$.

Expanding the definition of the first result gives us:

\[ \forall d \in R_A. \  \lambda i \in W. \  \llbracket \Gamma \vdash e : A \to B \rrbracket(d^*(i))(d(i)) \in R_B \]

Let $d = \lambda i \in W. \llbracket \Gamma \vdash e' : A  \rrbracket(d^*(i)) \in R_A$. Then we have:

\[  \lambda i \in W. \  \llbracket \Gamma \vdash e : A \to B \rrbracket(d^*(i))( \llbracket \Gamma \vdash e' : A \rrbracket(d^*(i)) \in R_B \]
 
Using the definition of the Denotational Semantics for function application, we have:

\[\lambda i \in W. \  \llbracket \Gamma \vdash  e \ e' : B \rrbracket (d^*(i)) \in R_B\]

which is what we wanted to prove.
\end{proof}

To use our relation to prove $\por$ is not definable in PCF, we must first define a property that a logical relation may or may not satisfy:

\section{R-invariant}

When the denotation of an expression $e$ of type $A$ is \textbf{$R$-invariant}, there is an element in $R_A$ of the form $(\llbracket \Gamma \vdash e : A \rrbracket d^*, \dots , \llbracket \Gamma \vdash e : A \rrbracket d^*) \in R_A$

We can define this in general for an denotation $d \in \llbracket A \rrbracket$:

\vspace{0.5cm}

\begin{defn}{\citep{Streicher06}}Let $R$ be a logical relation of arity $W$. Then $d \in \llbracket A \rrbracket$ is called $R$-invariant if 

\[\delta_W(d) = \lambda i \in W. \ d \in R_A \]
\end{defn}

%Therefore we can also say that if $R$ is a logical relation of arity $W$ and $e$ is a closed term of type $A$ then the denotation of $e$ is $R$-invariant.

To understand how this works for $\lambda$ terms, we can prove the following as a corollary of the Main Lemma:

\vspace{0.5cm}

\begin{cor}{\citep{Streicher06}}
If $R$ is a logical relation of arity $W$ and $e$ is a $\lambda$ term of type $A$, then $\lambda i \in W. \ \llbracket e \rrbracket \in R_A$
\end{cor}

Therefore any element of the denotational model that is the denotation of a closed $\lambda$ term is $R$-invariant.

By proving another corollary of the Main Lemma, it can be proved that any $\lambda$ expression is $R$-invariant (which is done in \citep{Streicher06}). For non-closed expressions, we can prove that as long as the denotations of all the expressions in the substitution are $R$-invariant, then the denotation of the entire non closed expression is also $R$-invariant, as a corollary of the main lemma (see Lemma \ref{main2}):

\vspace{0.5cm}

\begin{cor}{\citep{Streicher06}}
Let $R$ be a logical relation on the Scott Model of arity $W$ and $\Gamma \vdash e : B$ be a $\lambda$ term. Then $\llbracket \Gamma \vdash e : B \rrbracket(d^*(i))$ is $R$-invariant whenever all $d \in d^*$ are.
\end{cor}

Now we know that all the denotations of $\lambda$ terms are $R$-invariant, so any \textbf{closed} PCF term that can be written as a $\lambda$ term will have  an $R$-invariant denotation. 

The syntax of PCF can be written as $\lambda$ terms in the following way, as previously described:

\begin{itemize}
\item{$z$}
\item{$\lambda x: \nat. \ s(x)$}
\item{$\lambda x: \nat, e_0 : A, e_S: A. \ \case(e,z \mapsto e_0,s(x) \mapsto e_S)$}
\item{$\lambda x : A. \ \lambda e: A . \ \fix x:A \ e $}
\end{itemize} 


so these $\lambda$ terms can be composed to create $\lambda$ terms for any closed PCF term, so any closed PCF term will be $R$-invariant.

Therefore, we want to ensure that all of the $\lambda$ terms describing PCF expressions are $R$-invariant. The most difficult expression to check is $R$-invariant is $\lambda x:A. \ \lambda e: A . \ \fix x:A \ e $. For this we require another property on logical relations:

\section{Admissible Logical Relations}

An admissible logical relation is the following:

\vspace{0.5cm}

\begin{defn}{\citep{Streicher06}}
A logical relation $R$ of arity $W$ is called \textbf{admissible} if $\delta_W(\bot) \in R_{Nat}$ and for all chains $d_1 \sqsubseteq d_2 \sqsubseteq \dots$, if  $d_i \in R_{\nat}$, then $\bigsqcup d_n \in R_{\nat}$.
\end{defn}

In \citep{Streicher06}, the above definition is defined for directed sets. If $R_{\nat}$ is closed under directed suprema (i.e.\ the least upper bound of a directed set), then this is the same as saying that if we form chains $d_0 \sqsubseteq d_1 \sqsubseteq \dots$ of $W$-tuples of elements in $\mathbb{N}_\bot$'s underlying set, if $d_i \in R_{\nat}$, then $\bigsqcup d_n \in R_{\nat}$. This is how we obtained our definition of admissible.

It is also stated in \citep{Streicher06} that for finite $W$, there are no non trivial chains of $\llbracket \nat \rrbracket^W$, so in these cases, we can just say the logical relation is admissible if $\delta_W(\bot) \in R_{Nat}$.  

To show that $\lambda x : A. \ \lambda e : A. \ \fix x : A. \ e : A$ is $R$-invariant, we prove the following theorem, from $\citep{Streicher06}$, which is modified for our definition of fixpoint and the definition of domains and admissible logical relations using chains. Therefore the proof is quite different from the original:

\vspace{0.5cm}

\begin{thm}{\citep{Streicher06}}\label{adm}
Let $R$ be an admissible logical relation of arity $W$. Then for all types $A$ we have:

\begin{enumerate}
\item{$\delta_W(\bot) \in R_A$ and $R_A$ for all chains $d_1 \sqsubseteq d_2 \sqsubseteq \dots$, if  $d_i \in R_A$, then $\bigsqcup d_n \in R_A$}
\item{The denotation of $\lambda x : A. \lambda e : A. \fix x : A. \ e : A$ is $R$-invariant}
\end{enumerate}
\end{thm}

\begin{proof}
We prove \emph{1.} by induction on types. For base type, Nat, our goal is the same as the definition of admissible and we know $R$ is admissible. 

For function types $A \to B$, using the inductive hypothesis, we know that $\delta_W(\bot) \in R_A$ and $\delta_W(\bot) \in R_B$ and that for $R_A$, for all chains $d_1 \sqsubseteq d_2 \sqsubseteq \dots$, if  $d_i \in R_A$, then $\bigsqcup d_n \in R_A$ and that this holds for $R_B$ too, as we assume they are admissible. 

We want to show that $\delta_W(\bot) \in R_{A \to B}$, which is the same as $\lambda i \in W. \bot \in R_{A \to B}$. As $A \to B$ is a function type, $\bot$ here is the function $\bot = \lambda x. \bot(x)$. Therefore we must show that $\forall d \in R_A. \lambda i \in W. (\lambda x. \bot(x))(d(i)) \in R_B$, which is the same as $\forall d \in R_A. \lambda i \in W. \bot \in R_B$.

Let $d \in R_A$. Then we must show $\lambda i \in W. \bot \in R_B$. This is the same as $\delta_W(\bot) \in R_B$, which we already have.

\vspace{0.5cm}

To show that the limit of chains of elements in  $R_{A \to B}$ is also in $R_{A \to B}$, we first assume we have a chain of functions $f_1 \sqsubseteq f_2 \sqsubseteq \dots$, where $f_m \in R_{A \to B}$ for every element in the chain. Expanding any one of these definitions gives us:

\[ \forall d \in R_A. \ \lambda i \in W. f_m(i)d(i) \in R_B\]

Let $d$ be a tuple of elements of $\llbracket A \rrbracket$'s underlying set that is in $R_A$. Then we have $\lambda i \in W. \ f_m(i)d(i) \in R_B$, so we can have a chain of elements of $R_B$ (by monotonicity). By the inductive hypothesis, we know that the limit of a chain formed of  any elements of $R_B$ is also in $R_B$. Therefore the element $\lambda i \in W. \ \bigsqcup f_n(i)d(i) \in R_B$ (by continuity).

Therefore $\forall d \in R_A. \ \lambda i \in W. \ \bigsqcup f_n(i)d(i) \in R_B$, so $\bigsqcup f_n \in R_{A \to B}$.

\emph{Note that this is a more general form of the function types case of the Chains Lemma (see Lemma \ref{chain})}

\vspace{0.5cm}

%%% NOW THE FIX CASE %%%%
To prove \emph{2}, we must show 

\[ \lambda i \in W. \ \llbracket \lambda x : A. \ \lambda e : A. \ \fix x : A. \ e : A \rrbracket \in R_{A \to A \to A} \]

By the definition of the denotational semantics, this is the same as:

\[ \lambda i \in W. \ (\lambda x' \in \llbracket A \rrbracket. \lambda e' \in \llbracket A \rrbracket. \ \llbracket x:A, e :A \vdash \fix x : A. \ e : A \rrbracket (x',e')) \in R_{A \to A \to A} \]

As this is a function type, we should have:

\[ \lambda i \in W. (\forall d_x \in R_A. \forall d_e \in R_A. \]
\[ \lambda i \in W. \ (\lambda x' \in \llbracket A \rrbracket. \lambda e' \in \llbracket A \rrbracket. \ \llbracket x:A, e :A \vdash \fix x : A. \ e : A \rrbracket (e',x'))(d_x(i))(d_e(i)) \in R_A \]

Therefore we assume that we have a logical relation $R_A$ of arity $W$ and that $d_x \in R_A$ and $d_e \in R_A$. Then we want to show:

\[ \lambda i \in W. \  \llbracket x:A, e :A \vdash \fix x : A. \ e : A \rrbracket (d_x(i),d_e(i)) \in R_A \]

Using the denotational semantics for the fixpoint case, we have:

\[ \lambda i \in W. \  \Fix (\lambda a \in \llbracket A \rrbracket. \llbracket x:A, e :A \vdash  e : A \rrbracket (d_x(i),d_e(i),a) \in R_A \]

Which reduces to the following by using the denotational semantics for variables:

\[ \lambda i \in W. \  \Fix (\lambda a \in \llbracket A \rrbracket . d_e(i)) \in R_A \]

By using the fixpoint theorem (Lemma \ref{fixpoint}), we know that this is the same as:

\[ \lambda i \in W. \ \bigsqcup (\lambda a \in \llbracket A \rrbracket . d_e(i))^n(\bot) \in R_A \]


So we prove that $\forall n. \lambda i \in W. (\lambda a \in \llbracket A \rrbracket . d_e(i))^n(\bot) \in R_A$, by induction on $n$.

When $n = 0$, we have $\lambda i \in W. \bot$, which we know is in $R_A$ from part \emph{1} of this proof.

When $n = n+1$, we know $\lambda i \in W. (\lambda a \in \llbracket A \rrbracket . d_e(i))^n(\bot) \in R_A$, as this is the inductive hypothesis. As all denotations of PCF functions are continuous, and continuous functions are monotone, we can have $\lambda i \in W. (\lambda a \in \llbracket A \rrbracket . d_e(i)) (\lambda a \in \llbracket A \rrbracket . d_e(i)^n(\bot)) = \lambda i \in W.  d_e(i) \in R_A$.

Therefore $\lambda i \in W. \bigsqcup (\lambda a \in \llbracket A \rrbracket . d_e(i))^n(\bot) \in R_A$ , so the whole fixpoint $\lambda$ term is $R$-invariant and we have completed the proof. 
 
\end{proof}

%\textcolor{red}{I need to prove something like this for case too? (as it's not a natural number anymore...)}

Now we have proved this, we can prove the following:

\vspace{0.5cm}

\begin{thm}{\citep{Streicher06}}\label{t13}
Let $R$ be an admissible logical relation such that the denotations of the terms:

\begin{itemize}
\item{$z$}
\item{$\lambda x: \nat. \ s(x)$}
\item{$\lambda x: \nat, e_0 : A, e_S: A. \ \case(e,z \mapsto e_0,s(x) \mapsto e_S)$}
\end{itemize}

are all $R$-invariant. Then all denotations of closed PCF terms are $R$-invariant.
\end{thm}

\begin{proof}
Assume we have an admissible logical relation $R$ and the given $\lambda$ terms have $R$-invariant denotations. Then we just need to show the denotation of $\lambda x : A. \lambda e : A. \fix x : A. \ e : A$, which we get by Theorem \ref{adm}. Therefore all denotations of closed PCF terms are $R$-invariant.
\end{proof}

\section{Admissible Logical Relations for Non-Definability of por}\label{exercise}

Now we have defined admissible logical relations and proved that closed PCF terms are $R$-invariant for any such logical relation, we give two logical relations, defined in \citep{Streicher06} which we will use to prove that $\por$ is non-definable in PCF.

The following two logical relations are specified by defining them at base types in the following way:

\[ (x,y,z) \in {R^1_{Nat}} = x \uparrow y \ \wedge \ z = x \sqcap y\]

\[ (x,y,z) \in {R^2_{Nat}} = x = \bot \ \vee \ y = \bot \ \vee \ z = \bot \ \vee \ x=y=z \]

\emph{(where $x \sqcap y$ is the greatest lower bound of $x$ and $y$ and $x \uparrow y = \exists z. \ x \sqsubseteq z \ \wedge \ y \sqsubseteq z$)}

At function type, the definitions will be as for any logical relation:

\[ f \in R_{A \to B}^1 = \forall d \in R_A^1. \ \lambda i \in W. f(i)(d(i)) \in R_B^1 \]

\[ f \in R_{A \to B}^2 = \forall d \in R_A^2. \ \lambda i \in W. f(i)(d(i)) \in R_B^2 \]

We want to show that the denotations of all closed PCF terms are invariant in them. We use Theorem \ref{adm}, so we must prove two things. First we prove that both logical relations are admissible. As $W = 3$, from the remark before, we just prove that $(\bot, \bot, \bot)$ is in each logical relation, to show this.

\paragraph{$R_1$ is admissible} For $R_1$, $z = \bot$, so $x \uparrow y$ and $\bot \sqcap \bot = z$, so $\delta_W(\bot)$ is in $R^1$.

%Given a chain of numbers/$\bot$s, in $R_1$, their limits must also be in $R_1$. \textcolor{red}{(trivial?)}

\paragraph{$R_2$ is admissible}

$\delta_W(\bot)$ is in $R^2$, as everything is equal to $\bot$.

%\textcolor{red}{Have written on paper all possible elements of each, should I put it here too?}

Secondly, we must show that the denotations of all the $\lambda$ definitions of the constructors (except $\fix$) are $R$-invariant for each relation. (This is left as exercise in \citep{Streicher06}, we prove it fully here):


\subsection{Constructors are in $R_1$}

\begin{lem}
$\llbracket z \rrbracket$ is $R^1$ invariant
\end{lem}

\vspace{0.25cm}

\begin{proof} We must show that $(0,0,0) \in R_{\nat}^1$. Let $z = 0$. Then $0 \sqsubseteq 0$, so $x \uparrow y$. As $x$ and $y$ are both $0$, their greatest lower bound will be too, so $x \bigsqcap y = 0 = z$. Therefore $(0,0,0) \in R_{\nat}^1$.
\end{proof}

\vspace{0.5cm}

\begin{lem}
$\llbracket \lambda x : \nat . \ s(x) \rrbracket$ is $R^1$ invariant.
\end{lem}

\vspace{0.25cm}

\begin{proof} 
We want to show 
 that  $(\lambda n \in \mathbb{N}_{\bot}. \ \llbracket x : \nat \vdash s(x) : \nat \rrbracket (n/x),\lambda n \in \mathbb{N}_{\bot}. \ \llbracket x : \nat \vdash s(x) : \nat \rrbracket (n/x), \lambda n \in \mathbb{N}_{\bot}. \ \llbracket x : \nat \vdash s(x) : \nat \rrbracket (n/x)) \in R_{\nat \to \nat}^1$. 
 
As these are functions of type $\nat \to \nat$, we must show that for any $(x,y,z) \in R_{\nat}^1$, that the denotations we obtain when these values replace $n$ in each of the functions are still related by the relation. If any of these values are $\bot$, the denotation will also be $\bot$. If any of $(x,y,z)$ are $n$, then the result of their denotation function will be $n+1$. For $x \uparrow y$ to be true, if either $x$ or $y$ are $n$ then the other one must also be $n$. If $z = x \sqcap y$, then if $x$ or $y$ is $n$, then $z$ must be $n$. The conditions in $R_{\nat}^1$ still be true if we add 1 to the value that is $n$, as  numbers can only be related to themselves or $\bot$, so it does not matter what number $n$ is.

Therefore, for any $(x,y,z) \in R_{\nat}^1$,

 \[((\lambda n \in \mathbb{N}_{\bot}. \ \llbracket x : \nat \vdash s(x) : \nat \rrbracket (n/x)) (x),\]
 \[(\lambda n \in \mathbb{N}_{\bot}. \ \llbracket x : \nat \vdash s(x) : \nat \rrbracket (n/x)) (y), \]
 \[(\lambda n \in \mathbb{N}_{\bot}. \ \llbracket x : \nat \vdash s(x) : \nat \rrbracket (n/x))(z))\]
 \[ \in R_{\nat}^1\]
 
as the $\bot$s stay the same and we add 1 to the $n$s, so all the properties are preserved. This means that the denotation of the $\lambda$ term for successor is $R^1$ invariant.
\end{proof}

\vspace{0.5cm}

\begin{lem}
$\llbracket \lambda e : \nat, x : \nat, e_0 : A, e_S : A. \case (e, z \mapsto e_0, s(x) \mapsto e_S) \rrbracket$ is $R^1$ invariant.
\end{lem}

\vspace{0.25cm}

\begin{proof}

 We want to show that 
 \[ \lambda i \in W. (\]
\[\lambda n \in \mathbb{N}_{\bot}, x' \in \mathbb{N}_{\bot}, e_0' \in \llbracket A \rrbracket ,  e_S' \in \llbracket A \rrbracket .\]
\[ \llbracket e : \nat, x : \nat, e_0 : A , e_S : A \vdash \case (e, z \mapsto e_0, s(x) \mapsto e_S) \rrbracket\]
\[ (n/e, x'/x, e_0'/e_0, e_S'/e_S))\].
\[ \in R_{\nat \to \nat \to A \to A \to A}^1\]

As this is a function type, we assume that we have $(x,y,z) \in R_{\nat}^1$. Then we can replace $n$ with $x$, $y$ and $z$. If any of these values were $\bot$, the resulting denotation is $\lambda x' \in \mathbb{N}_{\bot}, e_0' \in \llbracket A \rrbracket ,  e_S' \in \llbracket A \rrbracket . \bot$.

If it is $0$, then we have 

\[ \lambda x' \in \mathbb{N}_{\bot}, e_0' \in \llbracket A \rrbracket ,  e_S' \in \llbracket A \rrbracket .\] \[\llbracket e : \nat, x : \nat, e_0 : A , e_S : A \vdash e_0 \rrbracket\]
\[ (n/e, x'/x, e_0'/e_0, e_S'/e_S)\]

and for $e_S$ we have the same but we swap out the denotation of $e_0$ for the denotation of $e_S$.

The denotation of the function with $\bot$ will have less information than the other two functions, which are equally informative, so the properties in $(x,y,z) \in R_{\nat}^1$ will be preserved. Therefore the $\lambda$ term for $\case$ is $R^1$ invariant.
\end{proof}



\subsection{Constructors are in $R_2$}

\begin{lem}
$\llbracket z \rrbracket$ is $R^2$ invariant
\end{lem}

\vspace{0.25cm}

\begin{proof}
We want to show that $(0,0,0) \in R_{\nat}^2$, which we know because $0=0=0$.
\end{proof}

\vspace{0.5cm}

\begin{lem}
$\llbracket \lambda x . \ s(x) \rrbracket$ is $R^2$ invariant
\end{lem}

\vspace{0.25cm}

\begin{proof}
Given some $(x,y,z) \in R_{\nat}^2$, then if any of these values are $\bot$, then 

\[(\lambda n \in \mathbb{N}_{\bot}. \ \llbracket x : \nat \vdash s(x) : \nat \rrbracket (n/x))(\bot)\]

will also be $\bot$, so the successor term is in $R_{\nat}^2$.

If they are equal, then the denotation of the successor terms will either all be the same $n + 1$, or will all be $\bot$, so will also be in $R_{\nat}^2$.

Therefore for all $(x,y,z) \in R_{\nat}^2$ that are given as parameters to $\lambda i \in W. \llbracket \lambda x . \ s(x) \rrbracket$, the resulting denotations will also be in $R_{\nat}^2$, so $\lambda i \in W. \llbracket \lambda x . \ s(x) \rrbracket \in R_{\nat \to \nat}^2$, and $\llbracket \lambda x . \ s(x) \rrbracket$ is $R^2$ invariant.
\end{proof}


\vspace{0.5cm}

\begin{lem}
$\llbracket \lambda e : \nat, x : \nat, e_0 : A, e_S : A. \case (e, z \mapsto e_0, s(x) \mapsto e_S) \rrbracket$ is $R^2$ invariant.
\end{lem}

\begin{proof} For some $(x,y,z) \in R_{\nat}^2$, if one of them is equal to $\bot$, then the denotation of the case expression with it will be $\lambda x \in \nat, \ e_0 \in \llbracket A \rrbracket, e_S \in \llbracket A \rrbracket . \ \bot$, which is the function that does not terminate, so will be $\bot$ in $R_{\nat \to \nat \to A \to A \to A}^2$.

Otherwise, $x = y = z = n$, so the denotation of all the case expressions is either 

\[ \lambda x' \in \mathbb{N}_{\bot}, e_0' \in \llbracket A \rrbracket ,  e_S' \in \llbracket A \rrbracket .\] \[\llbracket e : \nat, x : \nat, e_0 : A , e_S : A \vdash e_0 \rrbracket\]
\[ (n/e, x'/x, e_0'/e_0, e_S'/e_S)\]

or 

\[ \lambda x' \in \mathbb{N}_{\bot}, e_0' \in \llbracket A \rrbracket ,  e_S' \in \llbracket A \rrbracket .\] \[\llbracket e : \nat, x : \nat, e_0 : A , e_S : A \vdash e_S \rrbracket\]
\[ (n/e, x'/x, e_0'/e_0, e_S'/e_S)\]

so they will all be the same function. Therefore the $\lambda$ term is $R^2$ invariant.

\end{proof}

Now we know that all PCF constructor $\lambda$ terms are all $R^1$ invariant, so by Theorem \ref{t13}, all denotations of closed PCF terms are $R^1$ invariant. 

\subsection{Stable PCF functions}

Now we can prove a lemma which says that all first order PCF terms are \emph{stable}, from which the non-definability of $\por$ is a corollary. Stable means that binary infima (greatest lower bounds) of consistent pairs are preserved (i.e.\ given $x$ and $y$ such that $x \uparrow y$, then if $z = x \sqcap y$, then $f(z) = f(x) \sqcap f(y)$):

\vspace{0.5cm}

\begin{lem}{\citep{Streicher06}}\label{stable}
For every expression $e$ of first order type, (i.e.\ of type $\nat \to \nat \to \dots \to \nat$, (for some finite number of $\nat$s)), we have:

\[ \llbracket e \rrbracket (x_1 \sqcap y_1) \dots (x_k \sqcap y_k)  =  \llbracket e \rrbracket(x_1) \dots (x_k) \sqcap \llbracket e \rrbracket(y_1) \dots (y_k)\]

for all $x^*$ and $y^* \in \llbracket \nat \times \dots \times \nat \rrbracket$, with $x^i \uparrow y^i$ for all $i = 1, \dots , k$ 
\end{lem}

We give the proof as in \citep{Streicher06}, explaining it in more detail:

\begin{proof}
As we know that all the  $\lambda$ terms of the constructors are $R^1$ invariant and $R^1$ is admissible, then the denotation of any closed PCF term is $R^1$ invariant. This means for any function $\llbracket e \rrbracket$ that describes the denotation of a closed PCF term, and any $(x,y,z) \in R_{\nat}^1$, that

\[ \llbracket e \rrbracket (x) \uparrow \llbracket e \rrbracket(y) \ \wedge \ \llbracket e \rrbracket(z) = \llbracket e \rrbracket(x) \sqcap \llbracket e \rrbracket(y) \]

We know every $x^i \uparrow y^i$ so we have $(x^i, y^i, x^i \sqcap y^i) \in R_{\nat}^1$.

Applying the definition of $(\llbracket e \rrbracket , \llbracket e \rrbracket, \llbracket e \rrbracket) \in R_{\nat \to \dots \nat}^1$ we get:

\[ (\llbracket e \rrbracket(x^*), \llbracket e \rrbracket(y^*), \llbracket e \rrbracket (x^* \sqcap y^*)) \in R_{\nat}^1 \]

Therefore we have $(\llbracket e \rrbracket(x_1, \dots ,x_n), \llbracket e \rrbracket(y_1, \dots ,y_n),\llbracket e \rrbracket(x_1 \sqcap y_1), \dots ,(x_n \sqcap y_n)) \in R_{\nat}^1$.

%\textcolor{red}{Assuming $x^* \sqcap y^* = (x_1 \sqcap y_1), \dots (x_n \sqcap y_n)$}

By the second part of $R_{\nat}^1$'s definition, we have:
\[ \llbracket e \rrbracket(x_1 \sqcap y_1), \dots ,(x_n \sqcap y_n) = \llbracket e \rrbracket(x_1, \dots x_n) \sqcap \llbracket e \rrbracket(y_1, \dots, y_n)\]
\end{proof}

Now we can use this to show the non definability of parallel or:

\vspace{0.5cm}

\begin{cor}{\citep{Streicher06}}
There are no PCF definable functions $f$ of type $\nat \to \nat \to \nat$ where:

\begin{itemize}
\item{$f \ 0 \ \bot = 0$}
\item{$f \ \bot \ 0 = 0$}
\item{$f \ 1 \ 1 = 1$}
\end{itemize}
\end{cor}

\begin{proof}{\citep{Streicher06}}
By contradiction. Assume there is a function $f$ that satisfies the above conditions and it is definable in PCF. Then $f \ 0 \ \bot = 0 = f \ \bot \ 0$, so using Lemma \ref{stable} we have
\[ f \ \bot \ \bot = f \ (0 \sqcap \bot) \ (\bot \sqcap 0) = f \ 0 \ \bot \sqcap f \ \bot \ 0 = 0 \sqcap 0 = 0 \]

However we also have $f \ 1 \ 1 = 1$ and by monotonicity of $f$ we should have $f \ \bot \ \bot \sqsubseteq f \ 1 \ 1$ which is $0 \sqsubseteq 1$ which is not in the relation for $\mathbb{N}_\bot$. Therefore we have a contradiction and there is no PCF definable $f$ satisfying the constraints.
\end{proof}

The constraints in the above corollary characterize parallel or, so we cannot define parallel or in our PCF.

We have not used $R^2$ yet. $R^2$ is another relation with which we can show that $\por$ is non-definable in 
PCF, as a corollary of Lemma 
 \ref{stable}.
\vspace{0.5cm}

\begin{cor}
Parallel Or is not definable in PCF 
\end{cor}

\begin{proof}{\citep{Streicher06}}
As all the $\lambda$-terms for PCF constructors are in $R^2$ and it is admissible, it is also the case that the denotation of any closed PCF term of type $\nat \to \nat \to \nat$ is $R^2$ invariant.  If this term is $f$, then we cannot satisfy all the por conditions (from the other corollary).

This is because $(\bot, 0, 1)$ and $(0, \bot, 1)$ are in $R_{\nat}^2$ and as both tuples contain $\bot$, we should have $f (\bot, 0, 1)(0, \bot, 1) \in R_{\nat}^2$ . However, the expansion of this is $(f \ \bot \ 0, f \ 0 \ \bot, f \ 1 \ 1) = (0,0,1)$, so it is not in $R_{\nat}^2$. Therefore we cannot define an $f$ that satisfies the constraints, so parallel or is not definable.
\end{proof}