\section{Lemmas for Main Lemma}
We also prove some other lemmas to help in the main lemma. The first covers the cases where $e$ does not terminate, to prove they are still in $\adeq$:

\subsection{Non-Termination Lemma}
\begin{lem}\label{infinite}
If $\llbracket e \rrbracket = \bot$ and $\vdash e : A$ and $e \mapsto^{\infty}$, then $e \in \adeq_A$
\end{lem}

\begin{proof}
By induction on types. The base case will be for the type $\nat$. We need to show that $\vdash e : \nat$ (which we already have as an assumption) and $\llbracket e \rrbracket = n \Rightarrow e \mapsto^* \underline{n}$. 

We can rewrite this as $\llbracket e \rrbracket \neq n \ \vee \ e \mapsto^* \underline{n}$

As we have $\llbracket e \rrbracket = \bot$ as an assumption and $\bot \notin \mathbb{N}$, we know that $\llbracket e \rrbracket \neq n$ for any $n \in \mathbb{N}$. Therefore $\llbracket e \rrbracket \neq n \ \vee \ e \mapsto^* \underline{n}$, so $\llbracket e \rrbracket = n \Rightarrow e \mapsto^* \underline{n}$ and $e \in \adeq_{\nat}$.  

For the inductive case, we need to show that $e \in \adeq_{A \to B}$, so we need $\vdash e : A \to B$ (which we have as an assumption) and $\forall e' \in \adeq_A. \ e \ e' \in \adeq_B$.

Let $e' \in \adeq_A$. As $\llbracket e \rrbracket = \bot$ is the bottom element of $\llbracket A \rrbracket \to \llbracket B \rrbracket$, this is the same as $\lambda a \in A. \bot_B$.  We can apply the inductive hypothesis to get $e \ e' \in \adeq_B$, as we know $\llbracket e \ e' \rrbracket = \bot_B$ and $\vdash e \ e' \in B$ (from the typing rule for function application with $e' \in Good_A$), so to do this, we just need to prove $e \ e' \mapsto^{\infty}$, which we prove by contradiction:

Assume $e \ e' \mapsto \underline{n}$, for some $n \in \mathbb{N}$. Then by correctness, we have $\llbracket e \ e' \rrbracket = n$. But $\llbracket
e \ e' \rrbracket = \bot_B$, so we have a contradiction. So $e \ e' \mapsto^{\infty}$.Now we apply the inductive hypothesis and get $e \ e' \in \adeq_B$. 

So for any $e' \in \adeq_A$, we have $e \ e' \in \adeq_B$.

Now we have proved the lemma for any type.
\end{proof}

We also prove the following lemma on substitutions:

\subsection{Substitution Lemma}

\begin{lem}\label{closedgamma}
If $\gamma \in \adeq_{Ctx}(\Gamma)$ and $\Gamma \vdash e : A$ then $\vdash [\gamma](e) : A$
\end{lem}

\begin{proof}
By induction on $\Gamma$. The base case is when $\Gamma$ is the empty context. We have $\gamma = \ <>$, so we need to prove $\vdash \ <>e : A$. The empty substitution does nothing, so this is the same as $\vdash e : A$, which we already have as an assumption.

The inductive case is where we have $\gamma \in \adeq_{Ctx}(\Gamma,x:A)$. Let $\gamma = (\gamma',e'/x)$ where $\gamma' \in \adeq_{Ctx}(\Gamma)$ and $e' \in  \adeq_A$. 

From $e' \in \adeq_A$, we have $\vdash e' : A$. We can use weakening on each variable in the context individually to get $\Gamma \vdash e' : A$. We use type substitution with this and the assumption $\Gamma, x:A \vdash e:A$ to get $\Gamma \vdash  [e'/x]e : A$.

Now we apply the inductive hypothesis of the theorem with $\gamma' \in \adeq_{Ctx}(\Gamma)$ to get $\vdash [\gamma'][e'/x'] e : A =  \ \vdash [\gamma]e : A$

Now we have proved the lemma for any type.
\end{proof}