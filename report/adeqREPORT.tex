\section{Other Proofs of Adequacy}\label{ch8}
Other proofs for Adequacy of PCF exist. Here we discuss some of the other approaches:

\subsection{Proof using "Computable Terms"}
The following definition of a computable term in PCF is given by \citep{Gunter92} and is adapted from \citep{Plotkin77}:

\vspace{0.5cm}

\begin{defn} A computable term is defined by the following statements:
\begin{enumerate}
\item{If $e$ is a closed PCF term of type $\nat$, then $e$ is computable if $\llbracket e \rrbracket = n \Rightarrow e \mapsto^* n$}
\item{If $e$ is a closed PCF term of type $A \to B$, it is computable if whenever $e'$ is a closed computable term of type $A$, then $e \ e'$ is a closed computable term of type $B$}
\item{If $e$ is an open term with free variables $x_1, \dots, x_n$ of types $A_1, \dots A_n$ then it is computable if $[e_1/x_1, \dots e_n/x_n]e$ is computable when $\forall i. \ e_i$ are closed computable terms}
\end{enumerate}
\end{defn}  

As described by \citep{Gunter92}, by combining the second two parts of the definition of computable terms, we know that an expression $e$ is computable iff there is some substitution $\sigma$ of closed computable expressions for all the free variables of $e$ (ie.\ part 3 of the definition) and there are closed computable terms $e_1, \dots e_n$ that make $(\sigma e)e_1, \dots e_n$ a closed term of ground type, then  $(\sigma e)e_1, \dots e_n$ is computable (using part 2 of the definition).  

%get that $e : A \to B$ is computable only if  $\sigma e(e'_1, \dots e'_n)$ is (where $\forall i. \ e'_i :  A_i$ is closed computable and $\sigma $ is a substitution formed of closed computable terms). This is because starting from the open term $e$, using part 3 of the definition we know that everything in $\sigma$ is closed computable, so if $\sigma e$ is closed computable, then so is $e$. Then we use part 2 to prove $\sigma e$ is closed computable. We get $\sigma e (e'_1)$ is closed computable, then $(\sigma e (e'_1))(e'_2)$ is closed computable and so on until we have applied every argument and the whole result is closed computable, so $\sigma e$ is too.

%Therefore if open terms are substituted with a closed computable substitution are closed computable and the application of them to any closed computable terms will give a closed computable term as a result, then the open function term is also closed computable. 

This is used to prove terms of types other than base type are computable:

\vspace{0.5cm}

\begin{thm}
Every PCF term is computable
\end{thm}

%summarize this, just do fixpoint case!
\begin{proof}
By induction on $e$. We show just a couple of cases here, for $\case$ and $\fix$. The rest of the cases are given in \citep{Gunter92}.

For $\case$, we use the statement above and prove $\sigma \case(e, z \to e_0, s(v) \to e_S)(e_1, \dots e_n)$ is a closed computable expression of ground type (where $(e_1, \dots e_n)$ are all closed  computable and $\sigma$ contains only closed computable terms) to prove all possible $\case$ terms are computable. (This also works in closed $\case$ terms, where $\sigma$ is just the empty substitution).  

Assume $\llbracket  \sigma \case(e, z \to e_0, s(v) \to e_S)(e_1, \dots e_n) \rrbracket = \llbracket v \rrbracket$ for some $v$. Therefore it cannot be that $\llbracket \sigma e \rrbracket = \bot$, so there are two cases.

If $\llbracket \sigma e \rrbracket = n$, for some $n > 0$ then the entire denotation is $\llbracket \sigma e_S (e_1, \dots e_n) \rrbracket$ and by the inductive hypothesis,  $\sigma e_S (e_1, \dots e_n) \rrbracket \mapsto v$. By the inductive hypothesis we also have $\sigma e \mapsto n$. By using the congruence rule and evaluation rule in the operational semantics, we get from this that $\sigma \case(e, z \to e_0, s(v) \to e_S)(e_1, \dots e_n) \mapsto v$.

The false case is similar, but instead we use the induction hypothesis on $e$ to get $\sigma e \mapsto 0$ and $\sigma e_0 \mapsto v$.
%Function application just follows from the inductive hypothesis (on $\sigma e$ and $\sigma e'$ for open terms) and part 2 of the definition.

%For $\lambda$ abstraction, we prove $(\lambda x:A. \ \sigma e)(e_1 \dots e_n)$ is computable if $e'_1 \dots e_n$ are closed computable and all the free variables of $e$ have been substituted by closed computable terms in $\sigma$, by using parts $2$ and $3$ of the definition, so   any $\lambda x:A. \ e$ is closed computable.

Fixpoint is the hardest case. We want to prove $(\fix x:A. \ \sigma e)(e_1 \dots e_n)$ is computable if $e_1 \dots e_n$ are closed computable and all the free variables of $e$ have been substituted. To do this we use a syntactic approximation described and proved in \citep{Gunter92} by the following lemma:

\begin{lem}\label{fix}
Let $e = \fix x : A. \ e' : A$ be a well-typed expression and define:

\[e^0 = \fix x : A. \ x \]
\[ e^{k+1} = (\lambda x : A. e')(e^k) \]

Then 

\begin{enumerate}
\item{$\llbracket e \rrbracket = \bigsqcup_k \llbracket e^k \rrbracket$}
\item{If $e^k \ e_1, \dots e_n \mapsto v$ for some $V$ and $e$ is of ground type, then $e \ e_1, \dots e_n \mapsto v$} 
\end{enumerate}  
\end{lem}

So we now want to first prove $\sigma e^k (e'_1 \dots e'_n)$ is closed computable.

We prove this by induction on $k$. $\llbracket e^0 \rrbracket = \bot$, so there is no value that $e^0$ evaluates to, so computability is vacuously true.

For the inductive case, we assume $\sigma M^{k-1}$ is closed computable, and then evaluate $\sigma M^k$ in the operational semantics to get $[\sigma, \sigma M^{k-1}/x]e$.

By the inductive hypothesis, we know $\sigma  M^{k-1}$ and $e$ are closed computable, so    $[\sigma, \sigma M^{k-1}/x]e$ is too.

Now we know $\sigma M^k$ is computable, we must show $\sigma M^k (e_1, \dots, e_n)$ is too. The denotation of $M^k$ is $\bigsqcup M^k$, so we must prove $\llbracket \sigma e^k (e_1, \dots, e_n)\rrbracket = n$. We use Lemma \ref{fix} for this, with the fact that we now know $e^k$ is computable.
%continue with this.  
\end{proof}

\subsection{Differences from our Logical Relation}
The definition of computable could also be considered as a unary logical relation (or logical predicate), where PCF expressions satisfying the definition of computable are in the relation.

This predicate would only be defined on the syntax of PCF and not on the semantics, which makes it much more difficult to prove the fixpoint case, as we have to use the Lemmma \ref{fix} which is quite complicated to prove. \citep{Gunter92} uses another version of PCF with bounded recursion and an Unwinding Theorem (of which Lemma \ref{fix} is a corollary) to translate it back to our PCF, which adds complexity to their proof.  
%\section{PCF with fixpoint constant}
%
%We could have also defined PCF in a different way, by using constants to replace certain constructors in the language.
%
%For example, in our definition of PCF, we have terms $\succ \ e$, for some expression $e$. Here $\succ$ is a constructor, so $\succ$ itself is not a term, but once we attach an expression to it, it is a term, $succ \ e$.
%
%This is the same for zero, $\case$ and $\fix$.
%
%Alternatively, we could have defined these operations as constants, for example, $\succ : \nat \to \nat$ is a function in the language that has a pre defined definition that never changes. We then also define $zero : \nat$, $\case : \nat \to \nat \to \nat$ (just on Natural numbers) and $\fix : (A \to A) \to A$.
%
%This is the way PCF is defined in \citep{Streicher06}. Also Plotkin uses this definition, with the addition of a Boolean base type in \citep{Plotkin77}.
%
%\section{Proof using Binary Logical Relation}
%
%The proof of Adequacy given in \citep{Streicher06}, as well as using a different definition of PCF, also uses a different logical relation.

%Continue with material from other document.

%Just give the fixpoint case (and possibly successor)
