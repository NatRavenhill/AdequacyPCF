\subsection{Addition}

In functional languages, we usually define mathematical operators on numbers as recursive functions. For example addition is the following function:

\[\add  0 \ y = y \]
\[\add  s(n) \ y = s(\add  n \ y)\]

We can write this as a $\lambda$ abstraction:

\[ \add = \lambda x,y : \nat. \ case(x, z \mapsto y, s(v) \mapsto s(\add \ v \ y))\]

This is a recursive function, so to define it in PCF, it must be the fixpoint of some other function $A$:

\[ A = \lambda f: Nat \to Nat \to Nat. \ \lambda x,y : \nat. \ case(x, z \mapsto y, s(v) \mapsto s(f \ v \ y))\]

Therefore we can define this in PCF as:

\[ \add x \ y = (\fix f : \nat \to \nat \to \nat . \ A : \nat \to \nat \to \nat) \ x \ y  \].

When we try to evaluate this term, we get the following, by the evaluation rule for $\fix$:

$\fix f : \nat \to \nat \to \nat . \ A : \nat \to \nat \to \nat = [\fix f : \nat \to \nat \to \nat . \ A : \nat \to \nat \to \nat/f]A$. This expands to:

\[ \lambda x,y : \nat. \ case(x, z \mapsto y, s(v) \mapsto \]
\[s((\fix f: \nat \to \nat \to \nat . \ A : \nat \to \nat \to \nat) \ v \ y)) \]

Therefore expanding this infinitely gives all possible executions of the addition function.

%\end{document}

