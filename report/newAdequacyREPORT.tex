%Link back
Now that we have defined our operational semantics (in  Chapter \ref{ch4}) and our denotational semantics (in Chapter \ref{ch6}), we can relate them using the Adequacy Theorem: 

\vspace{1cm}
%Statement of thm
\begin{thm}\label{adeq}
If $\vdash e : \nat$ (i.e.\ $e$ is a closed term of type $\nat$), then $\forall n \in \mathbbm{N}. \ \llbracket e \rrbracket = n \Leftrightarrow e \mapsto^* \underline{n}$
\end{thm}

\vspace{0.5cm}

where $\llbracket e \rrbracket$ = $\llbracket \vdash e : \nat \rrbracket$ and $\underline{n}$ represents the numeral $n$, (as opposed to the actual natural number $n$).

\vspace{0.5cm}

%EXPLAIN WHY ONLY AT BASE TYPE.

%Correctness as in before
The theorem has two directions. We have mostly proved the backwards direction because of the Soundness proof we proved in the previous chapter (see Section \ref{sound}). Therefore we have proved half of Adequacy.

Therefore, the right to left direction of adequacy is a corollary of the soundness proof:

\vspace{0.5cm}

\begin{cor}
If $\vdash e : \nat$ and $\llbracket e \rrbracket = n$ then $e \mapsto^* \underline{n} \Rightarrow \llbracket e \rrbracket = n$
\end{cor}


\begin{proof}
We rewrite $e \mapsto^* \underline{n}$ as $e \mapsto e_0 \mapsto \dots \mapsto e_m \mapsto \underline{n}$ , for any $m \geq 0$. Applying soundness to these evaluations gives us $\llbracket e\rrbracket = \dots = \llbracket e_m \rrbracket = \llbracket \underline{n} \rrbracket$. Therefore we have $\llbracket  e \rrbracket = \llbracket \underline{n} \rrbracket$.

Now we need to prove $\forall n \in \mathbb{N}. \llbracket \underline{n} \rrbracket = n$, which we prove by induction on $n$:

If $n = 0$, then $\llbracket \underline{0} \rrbracket = \llbracket z \rrbracket = 0$, by the definition of the denotational semantics for zero.

%If $\underline{n} = zero$, then by the denotational semantics for $ zero$ we have $\llbracket e_n \rrbracket =  0$.

If $n = n + 1$, then $\llbracket \underline{n + 1} \rrbracket = \llbracket s(\underline{n}) \rrbracket$. By the inductive hypothesis we have $\llbracket \underline{n} \rrbracket = n$, so $\llbracket s(\underline{n}) \rrbracket = n + 1$. The definition of $\llbracket s(\underline{n}) \rrbracket$ is $\llbracket  \underline{n} \rrbracket + 1$. Therefore $\llbracket \underline{n + 1} \rrbracket = n + 1$.

%If $\underline{n} = s(v)$, then  in the rule, we have $\llbracket v \rrbracket$, which equals $v$ by the inductive hypothesis. Therefore, we use the case in the successor rule for a number and get $v + 1$.

Therefore $\forall n \in \mathbb{N}. \llbracket \underline{n} \rrbracket  = n$, so $\llbracket e \rrbracket =  \llbracket \underline{n} \rrbracket  = n$.

\end{proof}

To prove the forwards direction, we could naively try to prove it by induction on each expression. However,  although we only consider closed expressions of base type, if it has sub-expressions then they may not be closed or of type $\nat$, so we have no inductive hypothesis to provide in these cases.

Therefore, instead we use the Logical Relations approach that we described in Section \ref{log}.

%Binary Logical Relation definition
\section{Logical Relation}\label{const}
The logical relation we define is a binary logical relation between denotations of PCF expressions and closed expressions of PCF. It is based on a binary logical relation defined in \citep{Streicher06}. 

There are some differences between our logical relation and the one defined in \citep{Streicher06}. In our proof of the fixpoint case of the main lemma, we do not use the observational preorder that they use. Instead we use an Expansion Lemma (see Lemma \ref{exp} below). We also work with chains instead of directed sets.

Also, Streicher's logical relation is defined on a version of PCF in which  constants to replace certain constructors in the language.

For example, in our definition of PCF, we have terms $s(e)$, for some expression $e$. Here $s$ is a constructor, so $s$ itself is not a term, but once we attach an expression to it, it is a term, $s(e)$.

This is the same for zero, $\case$ and $\fix$.

Alternatively, we could have defined these operations as constants, for example, $\succc : \nat \to \nat$ is a function in the language that has a pre defined definition that never changes. We then also define zero $: \nat$, $\case : \nat \to \nat \to \nat$ (just on Natural numbers) and $\fix : (A \to A) \to A$. This is how PCF is defined in \citep{Streicher06}.
 
\subsection{Defintition of the logical relation}
The definition of the logical relation is the following:

\vspace{0.5cm}

\begin{defn}
We create a family of binary  relations $R_A \subseteq \llbracket A \rrbracket \times \{e \ | \ \vdash e : A \}$, by defining relations by induction on types for each type as follows:

\[ d  R_{\nat}  e \Leftrightarrow \forall n \in \mathbbm{N}. \ d = n \Rightarrow e \mapsto^* n\]

\[f  R_{A \to B}  e \Leftrightarrow \forall d \in \llbracket A \rrbracket. \ \forall e' \in \{ e \ | \ \vdash e : A\}. \ d R_A e' \Rightarrow f(d) R_B  e \ e'\]

\end{defn}

The forwards direction of Adequacy is given by the definition of the relation on expressions of type $\nat$.

As we previously described in section \ref{log}, to prove our actual property, we prove a more general Main Lemma on the logical relation first. Our Main Lemma for this relation will be the following:

\vspace{0.5cm}

\begin{lem} \label{main}
If $\Gamma = x_1 : A_1, \dots, x_n : A_n$, $\Gamma \vdash e : A$ and $ \ d_1 R_{A_1} e_1, \ d_2 R_{A_2} e_2, \dots , \  d_n   R_{A_n} e_n$, then

\[ \llbracket \Gamma \vdash e : A \rrbracket (d_1, \dots, d_n)  \ R_A \  [e_1/x_1, \dots, e_n/x_n]e \]
\end{lem}

In the lemma we are given an expression $e$ that is well typed in a typing context $\Gamma$ and some denotations $d_i$ that are related to some expressions $e_i$.

The denotation of that expression applied to substitutions $(d_1, \dots d_n)$ for all the  free variables in $\Gamma$ will be related to the expression $e$ with the expressions that correspond to each denotation (according to our assumption) substituted for the free variables in $e$.

We will prove this by induction on each possible expression $e$.

%substitution function
\subsection{Substitution Function}
In the Main Lemma, we make multiple substitutions in the expression $e$, but we only defined our substitution function (in Chapter \ref{safe}) on single substitutions $[e'/x]e$. Therefore we define substituting multiple variables in the following way, where $\gamma = [e_1/x_1, \dots e_n/x_n]$:

\begin{adjustwidth}{2cm}{}
$[\gamma](zero) = zero$
\end{adjustwidth}

\begin{minipage}{3.5in}
\begin{align*}
[\gamma](x) = 
  \begin{cases} 
           [\gamma](x) & \text{if } x \in dom(\gamma) \\
           x & otherwise 
  \end{cases}
\end{align*} 
\end{minipage}

This says that if $x$ is present in $\gamma$, (there is a substitution for it), replace $x$ with the result of the substitution in $\gamma$. Otherwise there is nothing to replace $x$ with, so we return the variable unaltered. 

\vspace{0.5cm}

\begin{adjustwidth}{2cm}{}

$[\gamma](s(e)) = s([\gamma] (e))$

$[\gamma](\case (e, z \mapsto e_0, s(v) \mapsto e_S)) = \case ([\gamma](e), z \mapsto [\gamma](e_0), s(v) \mapsto [\gamma](e_S))$

$[\gamma](e \ e') = ([\gamma] e) ([\gamma] e')$

$[\gamma] (\lambda x:A. e) = \lambda x:A. [\gamma] e$

$[\gamma] (\fix \ x:A. e) = \fix \ x:A. [\gamma] e$
\end{adjustwidth}

\vspace{0.5cm}

For a non empty $\gamma = e_1/x_1, \dots e_n/x_n$, we have:

\[ [e_1/x_1, \dots , e_n/x_n] e = [e_1/x_1]([e_2/x_2]( \dots [e_n/x_n] e) \]

If we add a variable to the substitution we can define this in the following way:
\[ [t/x][\gamma] e = [\gamma, t/x] e \]

%Lemmas we need for main lemma
\section{Lemmas for Main Lemma}
We are almost ready to prove the Main Lemma, but first we prove some lemmas that we will use in the proof:


\subsection{Bottom Element Lemma}

\vspace{0.25cm}

\begin{lem} \label{bot}
For any type $A$ and $\Gamma \vdash e: A$, $\bot \ R_A \ e$ 
\end{lem}

\begin{proof}
By induction on types. For $\nat$, we want to show $\forall n \in \mathbbm{N}. \ \bot  = n \Rightarrow e \mapsto^* n$. Because $\bot \notin \mathbbm{N}$, this statement is vacuously true.

For $R_{A \to B}$, we have that $\bot$ is the function $\lambda x : A. \bot$, so we want to show $\forall  d \in \llbracket A \rrbracket. \ \forall e' \in \{ e \ | \ \vdash e : A\}. \ d R_A e' \Rightarrow (\lambda x : A. \bot)(d) R_B  e \ e'$. Assume $d$ is an element of the domain representing type $A$ and $e'$ is an expression of type $A$ such that $d \ R_A e'$. We want to show that $\bot \ R_B \ e \ e'$ and by the inductive hypothesis, we know for any $e : B$ that $\bot \ R_B \ e$, so we can just use this with $e \ e'$ to get our conclusion.
\end{proof}

\subsection{Expansion Lemma}\label{exp}

\vspace{0.25cm}

\begin{lem}\label{exp}
If $\Gamma \vdash e : A$ and $e \mapsto e'$ and $d R_A e'$ then $d R_A e$ 
\end{lem}

\begin{proof}
By induction on types. For base case we assume $d R_{\nat} e'$, so we have $\forall n \in \mathbbm{N}. \ d = n \Rightarrow e' \mapsto^* n$. Let $n = d$. Then, as $e \mapsto e'$ and $e' \mapsto^* n$, we know that $e \mapsto^*n$. Therefore, $d R_{\nat} e$.

For the inductive case, assume $e_0 \mapsto e'_0$ and $f \ R_{A \to B} \ e_0$, so we have $\forall a \in \llbracket A \rrbracket. \ \forall e' \in \{ e \ | \ \vdash e : A\}. \ a \ R_A \ e' \Rightarrow f(a) \ R_B \ e_0 \ e'$. Let $a$ be an element in the domain representing $A$ and $e_1$ be an expression of type $A$ such that $a R_A e_1$. Then $f(a) \ R_B \ e_0 \ e_1$.

By the inductive hypothesis we know that for any $e \mapsto e'$ for expressions of type $B$, that they are both related to the same denotation. Therefore we have $f(a) \ R_B \ e'_0 \ e_1$, so we know $\forall a \in \llbracket A \rrbracket. \ \forall e' \in \{ e \ | \ \vdash e : A\}. \ a \ R_A \ e' \Rightarrow f(a) \ R_B \ e'_0 \ e'$, so $f \ R_{A \to B} \ e'_0$. 
\end{proof}

\subsection{Chains Lemma}

This lemma was adapted from a similar lemma on directed sets from \citep{Streicher06}.

\vspace{0.25cm}

\begin{lem}\label{chain}
For an expression $e : A$ and a  chain $x_0 \sqsubseteq x_1 \sqsubseteq \dots$, if $x_n R_A e$, then $\bigsqcup x_n \ R_A \ e$
\end{lem}

\begin{proof}
By induction on types. For base type, $\nat$, we assume we have a chain of elements in $\mathbbm{N}_\bot$ such that $x_n \ R_{\nat} e$. Therefore for any element in the chain we know $\forall n \in \mathbbm{N} . \ x_n = n \Rightarrow e \mapsto^* n$. We have two cases depending on the values of $\bigsqcup x_n$:

\begin{enumerate}
\item{$\bigsqcup x_n = n$. Then we know that $e \mapsto^* \sqcup x_n$, as any $x_n R_{\nat} e$}
\item{$\bigsqcup x_n = \bot$. By Lemma \ref{bot} we know that $\bot R_{\nat} e$ for any $e$}
\end{enumerate}

The inductive case is for $R_{A \to B}$. Assume we have a chain $f_1 \sqsubseteq f_2 \sqsubseteq \dots$ of elements in $\llbracket A \rrbracket \to \llbracket B \rrbracket$ and $d \ R_A \ e'$ for some $d \in \llbracket A \rrbracket$ and $e : A$. Then we need to show $\bigsqcup f_n (d) \ R_B \ e \ e'$. By induction we know for any expression of type $B$ and a chain  $f_1(d) \sqsubseteq f_2(d) \sqsubseteq \dots$ of elements of $\llbracket B \rrbracket$, $\bigsqcup f_n(d)$ is related to that expression. Therefore we have $\bigsqcup f_n(d) \ R_B \ e \ e'$.
%for a chain of elements in $\llbracket A \rrbracket$, $ \bigsqcup x_n R_A e$ and also the functions are continuous,so $f (\bigsqcup x_n) = \bigsqcup (f x_n)$. Therefore we have a chain of elements in $B$ which are the upperBy induction, every element of a chain in $B$ is in
\end{proof}

\section{Main Lemma}
Finally, we can prove the Main Lemma (Lemma \ref{main} above):

\begin{proof}
By induction on the possible values of $e$.

\paragraph{Variables} For a variable $x_1 : A_1, \dots, x_n : A_n \vdash x : A$, assume for any $i = 1, \dots, n$ that $d_i R_{A_i} e_i$. Then we want to show:

\[ (\lambda (d_1, \dots, d_n) \in \llbracket \Gamma \rrbracket. \ \pi_i(d_1, \dots, d_n))(d_1, \dots, d_n) \ R_A \ [e_1/x_1, \dots, e_n/x_n]x \]

As we have $\Gamma \vdash x : A$, by the typing rule for variables we have $\Gamma(x) = A$, so $x \in dom(\Gamma)$ and $\exists i. \ d_i R_{A_i} e_i$ . Then on the right hand side we have $[e_i/x]x$. Therefore we want to show

\[ d_i R_{A_i} e_i \]

which we have as an assumption. 

%Therefore we know there is some $e'$ such that $e'/x \in \gamma$ and by our assumption there is some $d$ such that $d R_A e'$.

%On the right hand side, the substitution $[\gamma]$ will apply the substitution for $x$ and ignore everything else in $\gamma$. Therefore we want to show $d R_A e'$, which we already have as an assumption.

\paragraph{Zero} For $z : \nat$, we want to show:

\[ \llbracket \Gamma \vdash z : \nat \rrbracket (d_1, \dots, d_n) \ R_{\nat} \ [e_1/x_1, \dots, e_n/x_n]z \]

Expanding the definitions gives us $0 \ R_{\nat} \ z$, so we must show $\forall n \in \mathbbm{N}. 0 = n \Rightarrow z \mapsto^* n$, which is the case as $z$ reduces to $n$ in zero steps. Therefore $0 \ R_{\nat} \ z$.

\paragraph{Successor} For $x_1 : A_1, \dots, x_n : A_n \vdash s(e) : \nat$, assume for any $i = 1, \dots, n$ that $d_i R_{A_i} e_i$. Then we want to show:

\[ \llbracket \Gamma \vdash s(e) : \nat \rrbracket (d_1, \dots, d_n) \ R_{\nat} \ [e_1/x_1, \dots, e_n/x_n](s(e))\]

Which expands to $\forall n \in \mathbbm{N}. \ \llbracket \Gamma \vdash s(e) : \nat \rrbracket (d_1, \dots, d_n) = n \Rightarrow [e_1/x_1, \dots, e_n/x_n]s(e) \mapsto^* n $. By the inductive hypothesis, we know:

\[ \llbracket \Gamma \vdash e : \nat \rrbracket (d_1, \dots, d_n) \ R_{\nat} \ [e_1/x_1, \dots, e_n/x_n]e \]

This expands to $\forall n \in \mathbbm{N}. \ \llbracket \Gamma \vdash e : \nat \rrbracket (d_1, \dots, d_n) = n \Rightarrow [e_1/x_1, \dots, e_n/x_n]e \mapsto^* n $. (If the denotation of $e$ is $\bot$ then this is vacuously true.)

Therefore there will be two cases:

\begin{enumerate}
\item{If $\llbracket \Gamma \vdash e : \nat \rrbracket (d_1, \dots, d_n) = \bot$, then we must show $\bot R_{\nat} [e_1/x_1, \dots, e_n/x_n]s(e)$, which we get from Lemma \ref{bot}}%As there is no $n = \bot$ then the statement that gives $\bot R_{\nat} [x_1/t_1, \dots, x_n/t_n]s(e)$ is vacuously true.}
\item{If $\llbracket \Gamma \vdash e : \nat \rrbracket (d_1, \dots, d_n) = v$, then we must show $v+1 \ R_{\nat} \ [e_1/x_1, \dots, e_n/x_n]s(e)$. Let $n = v + 1$. From the inductive hypothesis we know that $[e_1/x_1, \dots, e_n/x_n]e \mapsto^* v$. Using the congruence evaluation rule for successor, we get $s([e_1/x_1, \dots, e_n/x_n]e) \mapsto s(v)$ and $s(v)$ is the same as $v+1$. Therefore we have $v + 1 \ R_{Nat} s(v)$, so if we use this with the congruence rule in Lemma \ref{exp}, we have  $v + 1 \ R_{Nat} s([e_1/x_1, \dots, e_n/x_n]e)$ }
\end{enumerate}

\paragraph{Case} For $x_1 : A_1, \dots, x_n : A_n \vdash \case (e,z \mapsto e_0, s(x) \mapsto e_S)$, assume for any $i = 1, \dots, n$ that $d_i R_{A_i} e_i$. Then we want to show:


\[ \llbracket \Gamma \vdash \case(e,z \mapsto e_0, s(x) \mapsto e_S) : A \rrbracket (d_1, \dots, d_n) \]
\[ R_A \]
\[ [e_1/x_1, \dots, e_n/x_n]\case (e, z \mapsto e_0, s(x) \mapsto e_S) \]

The result of the denotation depends on the value of $e$, so we have three cases:

\begin{enumerate}
\item{$\llbracket \Gamma \vdash e : \nat \rrbracket (d_1, \dots, d_n) = \bot$, then we must show $\bot \ R_A \ [e_1/x_1, \dots, e_n/x_n] \\ \case (z \mapsto e_0, s(x) \mapsto e_S)$, which we get by applying Lemma \ref{bot}.}
\item{$\llbracket \Gamma \vdash e : \nat \rrbracket (d_1, \dots, d_n) = 0$, then we want to show $\llbracket \Gamma \vdash e_0 : A \rrbracket (d_1, \dots, d_n) \ \\ R_A [e_1/x_1, \dots, e_n/x_n]\case (z \mapsto e_0, s(x) \mapsto e_S)$. As we have $\Gamma \vdash e_0 : A$ from the typing rule of case, we can get $\llbracket \Gamma \vdash e_0 : A \rrbracket (d_1, \dots, d_n) \ R_A [e_1/x_1, \dots, e_n/x_n]e_0$ by the induction hypothesis. We can now use this in Lemma \ref{exp}, with the Operational Semantics rule for case when the expression is zero, to get 

\[\llbracket \Gamma \vdash e_0 : A \rrbracket (d_1, \dots, d_n) \ R_A [e_1/x_1, \dots, e_n/x_n]\case (z \mapsto e_0, s(x) \mapsto e_S)\]
}
\item{$\llbracket \Gamma \vdash e : \nat \rrbracket (d_1, \dots, d_n) = n+1$ we want to show
\[ \llbracket \Gamma, x : \nat \vdash e_S : \nat \rrbracket(d_1, \dots , d_n, d) \]
\[\ R_A \]
\[ [e_1/x_1, \dots, e_n/x_n]\case (e, z \mapsto e_0, s(x) \mapsto e_S)\]

From the induction hypothesis we have 

\[ \llbracket \Gamma, x : \nat \vdash e_S : \nat \rrbracket(d_1, \dots, d_n, d) \ \ R_A \ [e_1/x_1, \dots, e_n/x_n, e/x]e_S\]

We can now use this in Lemma \ref{exp}, with the operational semantics rule for case when the expression is not zero to get 

\[ \llbracket \Gamma, x : \nat \vdash e_S : \nat \rrbracket(d_1, \dots , d_n, d) \]
\[ R_A \]
\[ [e_1/x_1, \dots, e_n/x_n]\case (e, z \mapsto e_0, s(x) \mapsto e_S)\].


}
\end{enumerate}

\paragraph{Application} For $x_1 : A_1, \dots, x_n : A_n \vdash e \ e' : B$, assume for any $i = 1, \dots, n$ that $d_i R_{A_i} e_i$. Then we want to show:

\[ \llbracket \Gamma \vdash  e \ e' : B \rrbracket (d_1, \dots, d_n) \]
\[ R_B \]
\[ [e_1/x_1, \dots, e_n/x_n]e \ e' \]

Using the Denotational Semantics and substitution function we can rewrite this as:

\[ \llbracket \Gamma \vdash  e : A \to B \rrbracket (d_1, \dots, d_n) (\llbracket \Gamma \vdash e' : B \rrbracket (d_1, \dots, d_n))\]
\[ \ R_B \]
\[ ([e_1/x_1, \dots, e_n/x_n]e) ([e_1/x_1, \dots, e_n/x_n] e') \]

By the inductive hypothesis we have

\[\llbracket \Gamma \vdash  e : A \to B \rrbracket (d_1, \dots, d_n) \ R_{A \to B} \ [e_1/x_1, \dots, e_n/x_n]e\]


Expanding this gives us $ \forall d \in \llbracket A \rrbracket. \ \forall e' \in \{ e \ | \ \vdash e : A\}. \ d R_A e' \Rightarrow \llbracket \Gamma \vdash  e : A \to B \rrbracket (d_1, \dots, d_n) (d) \  R_B \ [e_1/x_1, \dots, e_n/x_n]e \ e'$. Let $d = \llbracket \Gamma \vdash e' : B \rrbracket (d_1, \dots, d_n)$ and $e' = [e_1/x_1, \dots, e_n/x_n]e'$. Then we have 

\[\llbracket \Gamma \vdash  e : A \to B \rrbracket (d_1, \dots, d_n) (\llbracket \Gamma \vdash e' : B \rrbracket (d_1, \dots, d_n) \ R_B \ ([e_1/x_1, \dots, e_n/x_n]e) ([e_1/x_1, \dots, e_n/x_n] e')\]

\paragraph{$\lambda$-Abstraction} 
For $x_1 : A_1, \dots, x_n : A_n \vdash \lambda x : A. \ e : A \to B$, assume for any $i = 1, \dots, n$ that $d_i R_{A_i} e_i$. Then we want to show:

\[ \llbracket \Gamma \vdash \lambda x : A. \ e : A \to B \rrbracket (d_1, \dots, d_n) \ R_{A \to B} \ [e_1/x_1, \dots, e_n/x_n] (\lambda x : A. \ e) \]


%\[ \lambda a \in \llbracket A \rrbracket. \ \llbracket \Gamma, x : A \vdash e : B \rrbracket(\gamma, a/x) \ R_{A \to B} [\gamma] (\lambda x : A. \ e) \]

Expanding the definition of the logical relation gives us

\[ \forall d \in \llbracket A \rrbracket. \ \forall e' \in \{ e \ | \ \vdash e : A\}. \ d R_A e' \Rightarrow\]
\[ \llbracket \Gamma ,x:A \vdash e : B \rrbracket(d_1, \dots, d_n) (d) \ R_B \  [x_1/t_1, \dots, x_n/t_n](\lambda x:A. \ e) \ e'\]

Let $d$ be an element of the domain of type $A$ and $e'$ be an expression of type $A$ such that $d R_A  e'$.

Then by the using the denotational semantics for $\lambda$ abstraction, we want to show:

\[ (\lambda d \in \llbracket A \rrbracket. \ \llbracket \Gamma, x :A \vdash e : B \rrbracket (d_1, \dots, d_n,d))(d) \ R_B \ [e_1/x_1, \dots, e_n/x_n] (\lambda x:A. \ e) \ e'\]

which is the same as:

\[ \llbracket \Gamma, x :A \vdash e : B \rrbracket (d_1, \dots, d_n, d) \ R_B \ [e_1/x_1, \dots, e_n/x_n, e'/x] \ e \]

which we get by the inductive hypothesis, because from the evaluation rule we have \[[e_1/x_1, \dots, e_n/x_n] (\lambda x:A. \ e) \ e'  \mapsto [e'/x][e_1/x_1, \dots, e_n/x_n]e = [e_1/x_1, \dots, e_n/x_n, e'/x]e\]

 so we can use Lemma \ref{exp}. 

\paragraph{Fixpoint}
For $x_1 : A_1, \dots, x_n : A_n \vdash \fix x:A. \ e : A$, assume for any $i = 1, \dots, n$ that $d_i R_{A_i} e_i$. Then we want to show:

\[ \llbracket \Gamma \vdash \fix x:A. \ e : A \rrbracket (d_1, \dots, d_n) \ R_A \ [e_1/x_1, \dots, e_n/x_n] (\fix x:A. \ e : A) \]

The denotation of the left hand side is $\Fix( \lambda a \in \llbracket A \rrbracket. \  \llbracket \Gamma, x :A \vdash e : A \rrbracket(d_1, \dots, d_n, a))$.

By the fixpoint theorem, we know this is the same as 
\[\bigsqcup_n (\lambda a \in \llbracket A \rrbracket. \  \llbracket \Gamma, x :A \vdash e : A \rrbracket(d_1, \dots, d_n, a))^n(\bot)\]


If we can prove that \[\forall n \in \mathbbm{N}. \ (\lambda a \in \llbracket A \rrbracket. \ \llbracket \Gamma, x : A \vdash \ e : A \rrbracket(d_1 \dots d_n, a))^n(\bot) \ R_A \ [e_1/x_1, \dots e_n/x_n]\fix x : A. \ e : A\]

then we can use Lemma \ref{chain}, to get the above result.

We prove this by induction on $n$. When $n = 0$, we need to show $\bot \ R_A \ [e_1/x_1, \dots e_n/x_n]\fix x : A. \ e : A$, for which we just use Lemma \ref{bot}.

For the inductive case, we must show 

\[(\lambda a \in \llbracket A \rrbracket. \ \llbracket \Gamma, x : A \vdash e : A \rrbracket(d_1 \dots d_n,a))^{n+1}(\bot) \ R_A \ [e_1/x_1, \dots e_n/x_n]\fix x : A. \ e : A\]

We can rewrite the left hand side as $(\lambda a \in \llbracket A \rrbracket. \ \llbracket \Gamma, x : A \vdash e : A \rrbracket(d_1 \dots d_n,a))(\lambda a \in \llbracket A \rrbracket. \ \llbracket \Gamma, x : A \vdash e : A \rrbracket(d_1 \dots d_n,a))^n(\bot)) $, which is the same as:

\[ \llbracket \Gamma, x : A \vdash e : A \rrbracket(d_1 \dots d_n,(\lambda a \in \llbracket A \rrbracket. \ \llbracket \Gamma, x : A \vdash e : A \rrbracket(d_1 \dots d_n,a))^n(\bot)))\]

We know that 

\[(\lambda a \in \llbracket A \rrbracket. \ \llbracket \Gamma, x : A \vdash e : A \rrbracket(d_1 \dots d_n,a))^n(\bot))) \ R_A \ [e_1/x_1, \dots e_n/x_n]\fix x : A. \ e : A\]

by the inductive hypothesis, so we can use this as an assumption in the inductive hypothesis of the Main Lemma to get:

\[ \llbracket \Gamma, x : A \vdash e : A \rrbracket(d_1 \dots d_n,(\lambda a \in \llbracket A \rrbracket. \ \llbracket \Gamma, x : A \vdash e : A \rrbracket(d_1 \dots d_n,a))^n(\bot)))\]
\[ R_A \]
\[ [e_1/x_1, \dots e_n/x_n, [e_1/x_1, \dots e_n/x_n]\fix x : A. \ e : A/x]e \]

because 
\[[e_1/x_1, \dots e_n/x_n]\fix x : A. \ e : A \]

\[ = \fix  x : A. [e_1/x_1, \dots e_n/x_n]e : A \]

\[\mapsto [\fix x:A. [e_1/x_1, \dots e_n/x_n]e/x][e_1/x_1, \dots e_n/x_n] e : A\]

\[ = [[e_1/x_1, \dots e_n/x_n] \fix x:A. e/x] [e_1/x_1, \dots e_n/x_n] e : A\]

\[ = [e_1/x_1, \dots e_n/x_n, [e_1/x_1, \dots e_n/x_n] \fix x:A. e/x] e : A \]

We can the use Lemma \ref{exp} with this, to get 
\[(\lambda a \in \llbracket A \rrbracket. \ \llbracket \Gamma, x : A \vdash e : A \rrbracket(d_1 \dots d_n,a))^{n+1}(\bot) \ R_A \ [e_1/x_1, \dots e_n/x_n]\fix x : A. \ e : A\]

Therefore $\forall n \in \mathbbm{N}. \ (\lambda a \in \llbracket A \rrbracket. \ \llbracket \Gamma, x : A \vdash \ e : A \rrbracket(d_1 \dots d_n, a))^n(\bot) \ R_A \ [e_1/x_1, \dots e_n/x_n]\fix x : A. \ e : A$, so the Main Lemma holds for fixpoint.



%[\fix x:A. e : A/x][e_1/x_1, \dots e_n/x_n]\fix x : A. \ e : A $

%Then we use Lemma \ref{exp} with $[x_1/t_n, \dots, x_n/t_n] \fix x:A. \ e :A \mapsto 

 
%Using the fixpoint constant (see \ref{const}) , we can rewrite this as: 

%\[ \llbracket \Gamma \vdash \FIX \ (\lambda x:A. \ e) : A \rrbracket (d_1, \dots, d_n) \ R_A \ [x_1/t_1, \dots, x_n/t_n] (\FIX \ (\lambda x:A. \ e) : A) \]

%As $\FIX \ (\lambda x:A. \ e)$ is a function application, by its typing rule we have $\Gamma \vdash \lambda x:A . \ e : A \to A$, so we can use the inductive hypothesis to get:

%\[ \llbracket \Gamma \vdash \lambda x:A. \ e \rrbracket(d_1, \dots, d_n) \ R_{A \to A} \ [x_1/t_1, \dots, x_n/t_n](\lambda x :A. \ e)\]

%Then, we use Lemma \ref{fix} with $f = \llbracket \Gamma \vdash \lambda x : A . \ e \rrbracket (d_1, \dots, d_n)$ and $e = [x_1/t_1, \dots, x_n/t_n](\lambda x :A. e)$. This gives us $\llbracket \FIX \rrbracket \llbracket \Gamma \vdash \lambda x : A . \ e \rrbracket (d_1, \dots, d_n) \ R_A \ \FIX \ [x_1/t_1, \dots, x_n/t_n](\lambda x :A. e)$.

%Using the denotational semantics for function application, we have:

%\[ \llbracket \Gamma \vdash \FIX \ \lambda x : A . \ e \rrbracket (d_1, \dots, d_n) \ R_A \ [x_1/t_1, \dots, x_n/t_n]\FIX (\lambda x : A . \ e)\].




%Which we can rewrite as:
%
%\[ fix(\lambda a \in \llbracket A \rrbracket. \ \llbracket \Gamma, x:A \vdash e:A \rrbracket (\gamma, a/x)) \ R_A \ \fix x:A. \ [\gamma]e : A \]
%
%We want to use Lemma \ref{fix}, so we need to show:
%
%\textcolor{red}{\[ \lambda a \in \llbracket A \rrbracket. \ \llbracket \Gamma, x:A \vdash e:A \rrbracket (\gamma, a/x) \ R_{A \to A} \ [\gamma]e \]}
%
%\textcolor{red}{But we don't know that $[\gamma]e$ is a function!} 
%
%which expands to:
%
%\[\forall d \in \llbracket A \rrbracket. \ \forall e' \in \{ e \ | \ \vdash e : A\}. \ d R_A e' \Rightarrow \llbracket \Gamma, x:A \vdash e:A \rrbracket (\gamma, d/x) R_B  [\gamma]e \ e'\]
%
%The induction hypothesis is:
%
%\[ \llbracket \Gamma, x : A \vdash e : A \rrbracket(\gamma, d/x) \ R_A \  [\gamma, e'/x] e \]
%
%







\end{proof}


%main lemma proof