Logical relations, developed in \citep{Tait67},\citep{Plotkin73},\citep{Statman85}, are a proof technique that is used for establishing properties that cannot be proved by structural induction on program terms alone, due to higher order constructions being present in the structure we are proving a property of.

They have been used to prove Strong Normalisation (i.e.\ that every expression terminates) of systems such as the Simply Typed $\lambda$ Calculus, Type Safety and Program Equivalence.

We construct two logical relations in our report:

\begin{itemize}
\item{A relation between program terms and their denotations (in Chapter \ref{Adequacy})}
\item{A relation between program denotations (in Chapter \ref{por})}
\end{itemize}

%\begin{lem}

%Let $\Gamma \vdash e : A$ where $\Gamma = \{x_1 : \theta_1, \dots x_m : \theta_m\}$ and $f = \llbracket \Gamma \vdash M : \theta \rrbracket$. Suppose $\{R_\theta\}$ is an $n$-ary logical relation and
%
%\[\gamma_i = (d_{i1} \dots d_{im}) \in \llbracket \theta_1 \rrbracket \times \dots \times \llbracket \theta_m \rrbracket\] 
%
%where $i = 1, \dots n$, are such that $R_{\theta_j}(d_{1j} \dots d_{nj})(1 \leq j \leq m)$. Then 
%
%\[R_\theta(f\gamma_1, \dots , f\gamma_n)\]  
%\end{lem}
%
%This says that for a well typed term $M$, that has a denotation equal to a function $f$, if we have a logical relation for types, then any possible set of substitutions for the variables in $\Gamma$ (of which there are $n$), that are in that term will be a tuple, $\gamma_i$, in the relation, and applying $f$ to each of these tuples gives another tuple that will be in the relation. The resulting tuple will contain the result of $f$ for each substitution.