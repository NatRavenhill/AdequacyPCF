
\begin{thm}
Every continuous function $f : X \to X$ has a least fixpoint, which is the limit of the chain $\bot \sqsubseteq f(\bot) \sqsubseteq f^2(\bot) \sqsubseteq \dots$
\end{thm}

\vspace{0.5cm}

%The chain is an $\omega$ chain, because the natural numbers have no upper bound, so we have an upper bound $\omega = f^n(\bot)$.

\begin{proof}
 Define the fixpoint function $fix(f) \equiv \sqcup f^i (\bot)$. This is the limit of the chain in the theorem. We know this limit exists because as $f$ is continuous, $(X, \sqsubseteq, \bot)$ must form a domain, and by the definition of domain, all chains of $X$ have a limit. 

%First we must prove that this limit exists, so $\forall i \in \mathbb{N} . \ f^i(\bot) \sqsubseteq f^n(\bot)$. We prove this by induction. If $i = 0$, then the only element of the chain is $f^0(\bot) = \bot$. $\bot$ is the least element of the chain, so we have $\bot  \sqsubseteq f^n(\bot)$. 

%The inductive hypothesis is $\sqcup f^i(\bot)$ exists. Applying $f$ to this gives us $f(\sqcup f^i(\bot))$. $f$ is continuous, so this is equal to $\sqcup f(f^i(\bot)) = \sqcup f^{i+1}(\bot) $

\paragraph{$\sqcup f^i (\bot)$ is a fixpoint}
For the limit to be a fixpoint we must have  $f( \sqcup f^i (\bot)) =  \sqcup f^i (\bot)$.  As $f$ is continuous, we have $f( \sqcup f^i (\bot)) = \sqcup f(f^i(\bot)) = \sqcup f^{i+1}(\bot)$. The chain formed by $f^{i+1}$ is $f(\bot)  \sqsubseteq f^2(\bot) \sqsubseteq \dots$. This is the same as our original chain, but without $\bot$ at the start. Because $X$ is a domain, we know that $X$ has a least element, so $\forall x \in X. \ \bot \sqsubseteq x$. Therefore $\bot$ has no effect on the limit because every element is higher than it, so removing $\bot$ will not change the limit. This means that $\sqcup f^{i+1} (\bot) = \sqcup f^i(\bot)$.

\paragraph{$\sqcup f^i (\bot)$ is the least fixpoint}
Let $x$ be an element of our chain such that $fix(x) = x$. Then for $\sqcup f^i (\bot)$ to be the least fixpoint, we must have $\sqcup f^i (\bot) \sqsubseteq x$, so $x$ is an upper bound that is higher than $\sqcup f^i (\bot)$. To prove this, first we prove $x$ is an upper bound, so  $\forall n. \ f^n(\bot) \sqsubseteq x$. We prove by this by induction on $n$:

if $n = 0$, then $f^0(\bot) \sqsubseteq x$, This is the same as $\bot \sqsubseteq x$, which is true because $\bot$ is the least element of the chain.

Our inductive hypothesis is $f^n(\bot) \sqsubseteq x$. As $f$ is continous, $f$ is monotone, so $f(f^n(\bot)) \sqsubseteq f(x) = f^{n+1}(\bot) \sqsubseteq x$. Therefore we know that for any element $f^n(\bot)$ in the chain, $f^n(\bot) \sqsubseteq x$. 

As $\sqcup f^i (\bot)$ is a least upper bound, we know that  $\forall x \in X. \  \forall n. \ (f^n(\bot) \sqsubseteq x) \Rightarrow \sqcup f^i (\bot) \sqsubseteq x$. We have just proved the left hand side of this, so we now have $\sqcup f^i (\bot) \sqsubseteq x$.

\vspace{0.5cm}

Now we have proved that $\sqcup f^i (\bot)$ is the least fixpoint of $f$.
\end{proof}